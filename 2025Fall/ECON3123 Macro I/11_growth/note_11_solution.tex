\documentclass[12pt]{article}

\usepackage[utf8]{inputenc}
\usepackage{geometry}
\geometry{a4paper,scale=0.75}
\linespread{1.5}
\usepackage{graphicx} 
\usepackage{float} 
\usepackage{subfig} 
\usepackage{enumerate}
\usepackage{enumitem}
\usepackage{amsmath}
\usepackage{array}
\usepackage{booktabs}
\usepackage{multirow}
\usepackage{amsfonts}
\usepackage[english]{babel}
\usepackage{amsthm}
\usepackage{dcolumn}
\usepackage{multicol}
\usepackage{stfloats}
\usepackage{lscape}
\usepackage[figuresright]{rotating}
\RequirePackage{pdflscape}
\usepackage[toc,page]{appendix}
\usepackage{geometry}
\usepackage{longtable}
\usepackage{comment}
\usepackage{xcolor}

% -------- enumerated sub-labels (a), (b), … --
\usepackage{enumitem}
\setlist[enumerate,1]{label=(\alph*),ref=\alph*}
% ---------------------------------------------

\usepackage{hyperref}
\hypersetup{hidelinks,
	colorlinks=true,
	allcolors=black,
	pdfstartview=Fit,
	breaklinks=true}
\usepackage{csquotes}
\usepackage{natbib}
\bibliographystyle{apalike}
\newtheorem{definition}{Definition}
\newtheorem{theorem}{Theorem}
\newtheorem{proposition}[theorem]{Proposition}
\newtheorem{lemma}[theorem]{Lemma}
\newtheorem{corollary}[theorem]{Corollary}
\newtheorem*{remark}{Remark}
\newtheorem{example}{Example}
\newtheorem{exercise}{Exercise}
\newtheorem{assumption}{Assumption}[section] % number within sections


\begin{document}

\begin{center}
    ECON 3123: Macroeconomic Theory I\\
    {\large \textbf{Tutorial Note 11: Growth}}\\
    Solution to Exercises\\
    Teaching Assistant: Harlly Zhou
\end{center}
\paragraph{Example 1}
(4) No. You can only take out $\log m$.

(5) No. You can only take out $m^{\frac{5}{6}}$. This is the case of Cobb Douglas with $\alpha+\beta<1$.

(6) Yes. 
\begin{align*}
    F(mK, mN_1, mN_2) &= \left(\frac{1}{3}(mN_1)^{\frac{1}{3}} + \frac{2}{3}(mN_2)^{\frac{1}{3}}\right)^2 (mK)^{\frac{1}{3}}\\
    &= \left[m^{\frac{1}{3}}\left(\frac{1}{3}N_1^{\frac{1}{3}} + \frac{2}{3}N_2^{\frac{1}{3}}\right)\right]^2 m^{\frac{1}{3}} K^{\frac{1}{3}}\\
    &= m^{\frac{2}{3}} \left(\frac{1}{3}N_1^{\frac{1}{3}} + \frac{2}{3}N_2^{\frac{1}{3}}\right)^2 m^{\frac{1}{3}} K^{\frac{1}{3}}\\
    &= m \left(\frac{1}{3}N_1^{\frac{1}{3}} + \frac{2}{3}N_2^{\frac{1}{3}}\right)^2 K^{\frac{1}{3}}\\
    &= m F(K, N_1, N_2).
\end{align*}

\paragraph{Exercises}
\begin{enumerate}[label=\arabic*.]
    \item At steady state, 
    \[\frac{K_t}{A_t N_t} = \frac{K_{t+1}}{A_{t+1} N_{t+1}}.\]
    Consider the growth rate of $\frac{K}{N}$. We have
    \begin{align*}
        \frac{K_{t+1}}{N_{t+1}} &= \frac{K_{t+1}}{A_{t+1}N_{t+1}} A_{t+1}\\
        &= \frac{K_t}{A_t N_t} A_{t+1}\\
        &= \frac{K_t}{N_t} \frac{A_{t+1}}{A_t}\\
        &= \frac{K_t}{N_t} (1+g_A),
    \end{align*}
    where the second line is by substituting the steady state condition. The same process applies for $\frac{Y}{N}$.
    Similarly, consider the growth rate of $K_t$.
    \begin{align*}
        K_{t+1} &= \frac{K_{t+1}}{A_{t+1}N_{t+1}} A_{t+1} N_{t+1}\\
        &= \frac{K_t}{A_t N_t} A_{t+1} N_{t+1}\\
        &= K_t \frac{A_{t+1}}{A_t} \frac{N_{t+1}}{N_t}\\
        &= K_t (1+g_A) (1+g_N).
    \end{align*}
    Taking logarithm gives you the growth rates.
    \item \begin{enumerate}[label=(\arabic*)]
        \item Note that
        \begin{align*}
            \frac{Y_{t+1}}{Y_t} = 1+g_Y &= \frac{A_{t+1}}{A_t} \left(\frac{K_{t+1}}{K_t}\right)^{\frac{1}{3}} \left(\frac{N_{t+1}}{N_t}\right)^{\frac{2}{3}}\\
            &= (1 + g_A) (1 + g_K)^{\frac{1}{3}} (1 + g_N)^{\frac{2}{3}}.
        \end{align*}
        Taking log approximation, we get
        \[g_Y = g_A + \frac{1}{3} g_K + \frac{2}{3} g_N.\]
        When $g_Y = g_K$, we have
        \[g_Y = g_K = \frac{3}{2}g_A + g_N.\]
        \item Dividing both sides of the LoM by $K_t$, we get
        \[\frac{K_{t+1}}{K_t} = 1 - \delta + \frac{I_t}{K_t}.\]
        Rearranging the terms, we get
        \[\frac{I}{K} = g_K + \delta.\]
        This means that the investment should compensate for the depreciation and catch up with the capital growth rate.
        \item 8\%.
    \end{enumerate}
    \item \begin{enumerate}[label=(\arabic*)]
        \item Rewriting the production function using per effective labour terms yields
        \[\hat{y} = \hat{k}^{1-\alpha}.\]
        At steady state,
        \[s (\hat{k}_{ss})^{1-\alpha} = (\delta + g_A + g_N)\hat{k}_{ss}.\]
        Solving the equation yields
        \[\hat{k}_{ss} = \left(\frac{s}{\delta + g_A + g_N}\right)^\frac{1}{\alpha}.\]
        \item The steady state consumption per effective labour is
        \[\hat{c}_{ss} = (1-s)\hat{k}_{ss}^{1-\alpha} = (1-s)\left(\frac{s}{\delta + g_A + g_N}\right)^\frac{1-\alpha}{\alpha}.\]
        The golden rate of saving should solve the following problem:
        \[\max_{s\in(0,1)}\,\, (1-s)\left(\frac{s}{\delta + g_A + g_N}\right)^\frac{1-\alpha}{\alpha}.\]
        Taking derivative with respect to $s$ yields the first-order condition for the maximization problem:
        \[-\left(\frac{s}{\delta+g_A+g_N}\right)^{\frac{1-\alpha}{\alpha}} + \frac{1-s}{\delta + g_A + g_N}\left(\frac{s}{\delta+g_A+g_N}\right)^{\frac{1-\alpha}{\alpha}-1}=0.\]
        Dividing both sides by $\left(\frac{s}{\delta+g_A+g_N}\right)^{\frac{1-\alpha}{\alpha}-1}$, we obtain
        \[\frac{1-s}{\delta + g_A + g_N} = \frac{s}{\delta+g_A+g_N}.\]
        Solving the equation yields:
        \[s_G = \frac{1}{2}.\]
        \item Note that $g_K = g_A + g_N$. So in period $t$, the growth rate is 5\%. After this, the growth rate is 10\%. The saving rate is lower than the golden rule. The increase in $g_A$ has no effect on $s_G$. 
    \end{enumerate}
\end{enumerate}



\end{document}