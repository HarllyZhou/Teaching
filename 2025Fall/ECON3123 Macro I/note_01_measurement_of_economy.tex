\documentclass[12pt]{article}

\usepackage[utf8]{inputenc}
\usepackage{geometry}
\geometry{a4paper,scale=0.75}
\linespread{1.5}
\usepackage{graphicx} 
\usepackage{float} 
\usepackage{subfig} 
\usepackage{enumerate}
\usepackage{enumitem}
\usepackage{amsmath}
\usepackage{array}
\usepackage{booktabs}
\usepackage{multirow}
\usepackage{amsfonts}
\usepackage[english]{babel}
\usepackage{amsthm}
\usepackage{dcolumn}
\usepackage{multicol}
\usepackage{stfloats}
\usepackage{lscape}
\usepackage[figuresright]{rotating}
\RequirePackage{pdflscape}
\usepackage[toc,page]{appendix}
\usepackage{geometry}
\usepackage{longtable}
\usepackage{comment}
\usepackage{xcolor}

% -------- enumerated sub-labels (a), (b), … --
\usepackage{enumitem}
\setlist[enumerate,1]{label=(\alph*),ref=\alph*}
% ---------------------------------------------

\usepackage{hyperref}
\hypersetup{hidelinks,
	colorlinks=true,
	allcolors=black,
	pdfstartview=Fit,
	breaklinks=true}
\usepackage{csquotes}
\usepackage{natbib}
\bibliographystyle{apalike}
\newtheorem{definition}{Definition}
\newtheorem{theorem}{Theorem}
\newtheorem{proposition}[theorem]{Proposition}
\newtheorem{lemma}[theorem]{Lemma}
\newtheorem{corollary}[theorem]{Corollary}
\newtheorem*{remark}{Remark}
\newtheorem{example}{Example}
\newtheorem{exercise}{Exercise}
\numberwithin{equation}{section}
\newtheorem{assumption}{Assumption}[section] % number within sections

\begin{document}

\begin{center}
    ECON 3123: Macroeconomic Theory I\\
    {\large \textbf{Tutorial Note 1: Measurement of Macroeconomy}}\\
    Teaching Assistant: Harlly Zhou
\end{center}

\subsection*{Output}
\paragraph{Calculation of GDP}
There are 3 definitions of GDP, each corresponding to a way to calculate it.
\begin{enumerate}[label=(\arabic*)]
    \item GDP is the market value of the \textbf{final} goods and services produced \textbf{in the economy} during a given period.
    \item[-] Intermediate goods and services are not considered in this way of calculation.
    \item[-] Foreign producers' production within the economy is counted into this economy's GDP.
    
    \item GDP is the sum of \textbf{value added} in the economy during a given period.
    \item[-] Value added $=$ Value of production $-$ Value of intermediate goods used in production.

    \item GDP is the sum of \textbf{incomes} in the economy during a given period.
\end{enumerate}

\begin{exercise}
    Consider the following NIPA:
\end{exercise}


\end{document}