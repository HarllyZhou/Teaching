\documentclass[12pt]{article}

\usepackage[utf8]{inputenc}
\usepackage{geometry}
\geometry{a4paper,scale=0.75}
\linespread{1.5}
\usepackage{graphicx} 
\usepackage{float} 
\usepackage{subcaption} 
\usepackage{enumerate}
\usepackage{enumitem}
\usepackage{amsmath}
\usepackage{array}
\usepackage{booktabs}
\usepackage{multirow}
\usepackage{amsfonts}
\usepackage[english]{babel}
\usepackage{amsthm}
\usepackage{dcolumn}
\usepackage{multicol}
\usepackage{stfloats}
\usepackage{lscape}
\usepackage[figuresright]{rotating}
\RequirePackage{pdflscape}
\usepackage[toc,page]{appendix}
\usepackage{geometry}
\usepackage{longtable}
\usepackage{comment}
\usepackage{xcolor}

% -------- enumerated sub-labels (a), (b), … --
\usepackage{enumitem}
\setlist[enumerate,1]{label=(\alph*),ref=\alph*}
% ---------------------------------------------

\usepackage{hyperref}
\hypersetup{hidelinks,
	colorlinks=true,
	allcolors=black,
	pdfstartview=Fit,
	breaklinks=true}
\usepackage{csquotes}
\usepackage{natbib}
\bibliographystyle{apalike}
\newtheorem{definition}{Definition}
\newtheorem{theorem}{Theorem}
\newtheorem{proposition}[theorem]{Proposition}
\newtheorem{lemma}[theorem]{Lemma}
\newtheorem{corollary}[theorem]{Corollary}
\newtheorem*{remark}{Remark}
\newtheorem{example}{Example}
\newtheorem{exercise}{Exercise}
\newtheorem{assumption}{Assumption}[section] % number within sections


\begin{document}

\begin{center}
    ECON 3123: Macroeconomic Theory I\\
    {\large \textbf{Tutorial Note 9: Financial Market and Goods Market in Open Economy}}\\
    Teaching Assistant: Harlly Zhou
\end{center}

\subsection*{Exchange Rate}
The \textbf{nominal exchange rate} is the price of domestic currency in terms of foreign currency, denoted by $E$. An \textbf{appreciation} of the domestic currency means that the domestic currency becomes more expensive relative to the foreign currency, i.e., $E$ increases. A \textbf{depreciation} of the domestic currency means that the domestic currency becomes cheaper relative to the foreign currency, i.e., $E$ decreases.
\begin{example}[Cross rates and triangular arbitrage]
    Suppose that the nominal exchange rate is 2.5 USD/GBP in year 0. One year later, this rate becomes 2.0 USD/GBP. In year 0, the nominal exchange rate between GBP and EUR is 1.25 EUR/GBP, and EUR appreciated by 10\% against USD in 1 year.
    \begin{enumerate}[label=(\arabic*)]
        \item What is the appreciation/depreciation rate of USD against GBP? What about GBP against USD?
        \vspace{60pt}
        \item What must the exchange rate between EUR and USD be to avoid arbitrage in year 0? Denote it as the price of EUR in terms of USD.
        \vspace{60pt}
        \item Suppose that the exchange rate between EUR and GBP is 1.1 \text{EUR/GBP} in year 1. Is there any arbitrage opportunity? Suppose that you are holding 1,000,000 GBP in year 1.
        \vspace{60pt} 
    \end{enumerate}
\end{example}

\begin{exercise}
    Assume you are a trader with Deutsche Bank. From the quote screen on your computer terminal, you notice that Dresdner Bank is quoting 0.7627 EUR/USD and Credit Suisse is offering 1.1806 CHF/USD. You learn that UBS is making a direct market between the Swiss franc and the euro, with a current EUR/CHF quote of 0.6395. Show how you can make a triangular arbitrage profit by trading at these prices. Assume you have \$5,000,000 with which to conduct the arbitrage. What happens if you initially sell dollars for Swiss francs (CHF)? What EUR/CHF exchange rate will eliminate triangular arbitrage?
\end{exercise}

\paragraph{International Parity Conditions}
No-arbitrage condition leads to a set of conditions that must be satisfied between currencies. These conditions are called \textbf{international parity conditions}.

We learnt in class the following \textbf{uncovered interest parity relation} (UIP):
\[1 + i_t = (1 + i^*_t) \frac{E_t}{E^e_{t+1}}.\]
Consider a \textbf{forward contract} which is an agreement to buy and sell foreign currencies in the future ($t+k$) at prices agreed upon today ($t$) denoted by $F_{t,t+k}$. Then we can predict $E^e_{t+1}$ by $F_{t,t+1}$. This leads to the \textbf{covered interest parity relation} (CIP):
\[1 + i_t = (1 + i^*_t) \frac{E_t}{F_{t,t+1}}.\]
What is \textit{covered} here? The exchange rate risk. By using a forward contract, the agent already locks the future exchange rate, which avoids uncertainty. However, in the UIP relation, this term is only one expectation, which essentially has no hedging effect.

\begin{example}
    Suppose that the 1-year interest rate is 1\% in USD and 3\% in GBP. The exchange rate is 1.5 USD/GBP. Suppose you want to buy or sell 1,000,000 GBP now. 
    \begin{enumerate}[label=(\arabic*)]
        \item What should be the 1-year forward rate to avoid arbitrage?
        \vspace{60pt}
        \item You predict that the actual exchange rate will be 1.5 USD/GBP in 1 year. What should you do?
        \vspace{60pt}
        \item You predict that the actual exchagne rate will be 1.4 USD/GBP in 1 year. What should you do?
        \vspace{60pt}
    \end{enumerate}
\end{example}

\paragraph{Real Exchange Rate} 
The \textbf{real exchange rate} is the price of domestic goods in terms of foreign goods, denoted by $\epsilon$:
\[\epsilon = E \frac{P}{P^*}.\]
The real exchange rate can be very different from the nominal exchange rate depending on the inflation rate in each country. Think about the following case: 
\begin{example}
    Consider the exchange rate $E = 7.8$ HKD/USD. Suppose that iPhone 20 is sold at 1,500 USD and 10,000 HKD. 
    \begin{enumerate}[label=(\arabic*)]
        \item What is the real exchange rate?
        \vspace{60pt}
        \item Where will you prefer to buy the iPhone 20? Can you make a profit? Assume that the phone does not depreciate.
        \vspace{60pt}
        \item At what real exchange rate will you be indifferent between buying it in the US and in Hong Kong?
        \vspace{60pt}
    \end{enumerate}
\end{example}
You can see that there will be arbitrage opportunity if the real exchange rate is not 1. This motivates another international parity condition: \textbf{purchsing power parity} (PPP). 

There are two versions of purchasing power parity. The first version is called the \textbf{absolute PPP}, which is directly from the concept of real interest rate:
\[\epsilon = 1,\,\,\,\text{or equivalently, }\,\,\, E = \frac{P^*}{P}.\]
The second version is called the \textbf{relative PPP}:
\[\frac{E^e_{t+1}}{E_t} = \frac{1+\pi^{*e}_{t+1}}{1+\pi^e_{t+1}}.\]
It is a good exercise to show this by yourself and it is not hard.

\begin{exercise}
    Show the relative PPP relation.
\end{exercise}

Recall that we have the Fisher equation:
\[1 + i_t = (1 + r_t)(1 + \pi^e_{t+1}).\]
If $r_t = r^*_t$, then we get the \textbf{international Fisher equation} (IFE):
\[\frac{1 + i^*_t}{1+i_t} = \frac{1+\pi^{*e}_{t+1}}{1+\pi^e_{t+1}}.\]

Combining PPP, IFE, and CIP, we have
\[\frac{E^e_{t+1}}{E_t} = \frac{1+\pi^{*e}_{t+1}}{1+\pi^e_{t+1}} = \frac{1 + i^*_t}{1+i_t} = \frac{F_{t,t+1}}{E_t}.\]
The resulting parity
\[\frac{E^e_{t+1}}{E_t} = \frac{F_{t,t+1}}{E_t}\]
is called the \textbf{forward expectation parity} (FEP).

\begin{exercise}
    Suppose that when you trade foreign currencies, there will be a transaction fee $T = \tau X$, where $X$ is your transaction amount. Which of the above relation(s) (\textit{i.e.,} UIP, CIP, PPP, IFE, FEP) cannot hold? Show your proof.
\end{exercise}

\subsection*{Goods Market Equilibrium and Trade Balance}
\paragraph{Demand for Domestic Goods}
In an open economy, the demand for domestic goods, $Z$, is given by
\[Z = C + I + G - \frac{IM}{\epsilon} + X,\]
where
\begin{itemize}
    \item $C$ is domestic consumption, a function of disposable income $Y-T$, where $T$ is tax minus transfer, denoted by $C(Y-T)$;
    \item $I$ is domestic investment, a function increasing in output $Y$ and decreasing in borrowing rate $r+x$ where $r$ is the real interest rate and $x$ is the risk premium, denoted by $I(Y, r+x)$;
    \item $G$ is government spending, subject to any specification, typically as constant, denoted by $\bar{G}$;
    \item $IM$ is the imports, a function increasing in domestic income $Y$ and increasing in the real exchagne rate $\epsilon$ (price of domestic goods in terms of foreign goods), denoted by $IM(Y,\epsilon)$;
    \item $X$ is the exports, a function increasing in foreign income $Y^*$ and decreasing in the real exchange rate $\epsilon$, denoted by $X(Y^*, \epsilon)$.
\end{itemize}
We say that there is a \textbf{trade surplus} if $X - \frac{IM}{\epsilon} > 0$ or a \textbf{trade deficit} if $X - \frac{IM}{\epsilon} < 0$.

The following diagrams illustrates how to construct the demand for domestic goods and how to determine trade balance.
\begin{enumerate}[label=(\arabic*)]
    \item The first step is to draw the \textbf{domestic demand} $DD$, which is the sum of consumption $C$, investment $I$, and government spending $G$. Figure \ref{fig:dd} displays domestic demand. 
\end{enumerate}
\begin{figure}[htp]
    \centering
    \includegraphics[width=0.45\textwidth]{dd.png}
    \caption{Step 1: Domestic demand}
    \label{fig:dd}
\end{figure}
\begin{enumerate}[label=(\arabic*),resume]
    \item The second step is to subtract import from the domestic demand $DD$. Note that $\frac{IM}{\epsilon}$ is an increasing function in $Y$, so when $Y$ becomes larger, the subtraction amount increases, which yields a flatter $AA$ curve. Figure \ref{fig:aa} displays this step. 
\end{enumerate}
\begin{figure}[htp]
    \centering
    \includegraphics[width=0.45\textwidth]{aa.png}
    \caption{Step 2: Subtracting imports}
    \label{fig:aa}
\end{figure}

\begin{enumerate}[label=(\arabic*),resume]
    \item The last step is to add exports to $AA$ curve. Since exports does not depend on domestic income $Y$, the resulting $ZZ$ curve is parallel shift of $AA$. The upper panel of figure \ref{fig:zztb} displays this step.
    \item After calculating $X - \frac{IM}{\epsilon}$, we get the net exports, drawn in the lower panel of figure \ref{fig:zztb}.
\end{enumerate}
\begin{figure}[htbp]
    \centering
    \begin{subfigure}[c]{0.4\textwidth}
      \centering
      \includegraphics[width=\linewidth]{zz.png}
    \end{subfigure}
    \\[0.01cm]
    \begin{subfigure}[c]{0.4\textwidth}
      \centering
      \includegraphics[width=\linewidth]{tbab.png}
    \end{subfigure}
    \caption{Step 3: Adding exports; Trade balance}
    \label{fig:zztb}
  \end{figure}
Figure \ref{fig:zztb} shows a quite clear correspondence between demand for domestic goods and net exports. 
\begin{itemize}
    \item At output level $Y$, the net export is positive so that the total demand for domestic goods is larger than total domestic demand. This is shown as the $a-b$ segment in the upper panel. This corresponds to a trade surplus $ab$ in the lower panel. 
    \item At output level $Y_{TB}$, the net export is zero so that the total demand for domestic goods is equal to the total domestic demand. This is shown as the intersection of $DD$ curve and the $ZZ$ curve. In the lower trade balance panel, this corresponds to the point where the solid black line crosses the horizontal axis, indicating zero net export.
\end{itemize}
From the graph, it is clear that any output level above $Y_{TB}$ leads to a trade deficit and that any output level below $Y_{TB}$ leads to a trade surplus.

\paragraph{Equilibrium in Goods Market}
Same as in the closed economy, the equilibrium condition is such that supply equals demand, \textit{i.e.,}
\[Z = Y.\]

\begin{figure}[htbp]
    \centering
    \begin{subfigure}[c]{0.4\textwidth}
      \centering
      \includegraphics[width=\linewidth]{eqm_open.png}
    \end{subfigure}
    \\[0.01cm]
    \begin{subfigure}[c]{0.4\textwidth}
      \centering
      \includegraphics[width=\linewidth]{eqm_tb.png}
    \end{subfigure}
    \caption{Goods Market Equilibrium: Not necessarily be trade balance}
    \label{fig:eqmtb}
\end{figure}

Graphically, this is to intersect the $ZZ$ curve with the 45-degree line. \textbf{It is important not to mix this goods market equilibrium condition and trade balance condition where $ZZ$ intersects $DD$.}

Figure \ref{fig:eqmtb} illustrates the goods market equilibrium together with the trade balance condition. When $ZZ$ intersects $DD$, there is trade balance, at which the output lebel is $Y = Y_{TB}$. However, the goods market equilibrium output $Y_{\text{eqm}}$ \footnote{This is denoted by $Y^*$ in the graph. This is my fault because I forgot the use of $Y^*$ as foreign output, and I am super lazy to change the graph... QAQ} is on the right side of $Y_{TB}$, which means that there is a trade deficit when the goods market is in its equilibrium.

The point $S$ in the upper panel where the 45-degree line intersects the $DD$ line is the closed-economy equilibrium.

\begin{exercise}
    Consider the following behavioral equation characterizing an open economy:
    \begin{align*}
        Y &= C + I + G - \frac{IM}{\epsilon} + X\\
        C &= c_0 + c_1 (Y- T)\,\,\, (c_0 > 0, 0 < c_1 < 1)\\
        I &= b_0 + b_1 Y\,\,\, (b_1 > 0)\\
        G &= \bar{G},\,\,\,\, T = \bar{T}\\
        IM &= m_1 Y \epsilon\,\,\, (0 < m_1 < 1)\\
        X &= x_1 Y^*\,\,\, (0 < x_1 < 1)\\
        Y^* &= \bar{Y^*},\,\,\,\, \epsilon = \bar{\epsilon}.
    \end{align*}
   Derive the condition for $m_1$ such that at goods market equilibrium the domestic output leads to a trade surplus.
\end{exercise}

\paragraph{Savings under open economy}
Consider the equilibrium demand:
\[Y = C + I + G - \frac{IM}{\epsilon} + X.\]
Subtracting $C$ and $T$ on both sides, we obtain
\[Y - T - C = I + (G - T) + NX.\]
Adding net income from abroad, $NI$ on both sides, we obtain
\[(Y + NI -T) - C = I + (G - T) + (NX + NI).\]
$NX+NI$ is the current account $CA$, and $(Y+NI-T)-C$ is the private saving. Then we get the $IS$ relation in an open economy:
\[CA = S + (T-G) - I.\]

The following are some textbook questions that may be useful for your review.
\begin{exercise}
    Chapter 18, Question 5 in Blanchard, Olivier (2021), \textit{Macroeconomics}, 8th ed., Pearson.
  \end{exercise}

  \begin{exercise}
    Chapter 18, Question 6 in Blanchard, Olivier (2021), \textit{Macroeconomics}, 8th ed., Pearson.
  \end{exercise}


  \begin{exercise}
    Chapter 18, Question 7 in Blanchard, Olivier (2021), \textit{Macroeconomics}, 8th ed., Pearson.
  \end{exercise}


  \begin{exercise}
    Chapter 18, Question 8 in Blanchard, Olivier (2021), \textit{Macroeconomics}, 8th ed., Pearson.
  \end{exercise}

\subsection*{Effect of Real Exchange Rate}
Recall that we assume that $IM = IM(Y,\epsilon)$ increases with $\epsilon$ and that $X = X(Y^*, \epsilon)$ decreases with $\epsilon$. With an increase in $\epsilon$, that is, an appreciation, $IM$ increases and $X$ decreases. However, when we consider the $ZZ$ curve, the effect is ambiguous, since $\frac{IM}{\epsilon}$ has both of its numerator and denominator increasing in $\epsilon$. Therefore, it is important to distinguish the condition that leads to different results.

\paragraph{Marshall-Lerner condition}
The \textbf{Marshall-Lerner condition} says that if the condition holds, then an \textit{depreciation} ($\epsilon$ decreases) should lead to an \textit{increase} in net export ($X - \frac{IM}{\epsilon}$). The following example helps you understand the derivation of Marshall-Lerner condition. Meanwhile, the appendix of the chapter in the textbook provides a special case where the initial net export is zero. If you find the example too difficult, then it should be okay if you can understand the appendix.
\begin{example}
    Consider an open economy with the following behavioral equations:
    \begin{align*}
        Y &= C + I + G - \frac{IM}{\epsilon} + X\\
        C &= c_0 + c_1 (Y- T)\,\,\, (c_0 > 0, 0 < c_1 < 1)\\
        I &= b_0 + b_1 Y\,\,\, (b_1 > 0)\\
        G &= T = 1\\
        IM &= m_1 Y + m_2 \epsilon\,\,\, (0 < m_1 < 1, m_2 > 0)\\
        X &= x_1 \frac{Y^*}{\epsilon}\,\,\, (0 < x_1 < 1).
    \end{align*}
    Suppose that $Y^* = Y$ all the time.
    \begin{enumerate}[label=(\arabic*)]
        \item Derive the equilibrium output $Y_{\text{eqm}}$ as a function of $\epsilon$. Note that you need to eliminate $Y^*$. 
        \vspace{60pt}
        \item Derive the general Marshall-Lerner condition first. Then substitute this economy into the condition. Show that $NX > -m_2$ under Marshall-Lerner condition.
        \vspace{60pt}
        \item Show that under certain parametric restriction, if $m_1 < x_1$, then a depreciation induces an increase in equilibrium output. 
        \vspace{60pt}
        \item Interpret the relationship between the two conditions:
        \begin{itemize}
            \item $NX > -m_2$;
            \item $m_1 < x_1$.
        \end{itemize}
    \end{enumerate}
\end{example}

\begin{example}
    Consider the world economy with only two open economies. Both of them are characterized by the same set of behavioral equations. Due to comparative advantage, each of them produces some goods that the other cannot produce, thus leading to trade. Consider the following set of behavioral equations:
    \begin{align*}
        Y &= C + I + G - \frac{IM}{\epsilon} + X\\
        C &= c_0 + c_1 (Y- T)\,\,\, (c_0 > 0, 0 < c_1 < 1)\\
        I &= b_0 + b_1 Y\,\,\, (b_1 > 0)\\
        G &= T = G^* = T^* =  1\\
        IM &= m_1 Y + m_2 \epsilon\,\,\, (0 < m_1 < 1, m_2 > 0)\\
        X &= x_1 \frac{Y^*}{\epsilon}\,\,\, (0 < x_1 < 1).
    \end{align*}
    \begin{enumerate}[label=(\arabic*)]
        \item Use a system of two equations with unknowns $Y$ and $Y^*$ to characterize the world economy equilibrium output.
        \vspace{60pt}
        \item Suppose that the depreciation rate is 0.5\%. As a result, in the new equilibrium, the domestic import decreases by 2\% and the export increases by 3\%. In the initial equilibrium, the ratio between export and import in terms of domestic goods is 2. Use the general Marshall-Lerner condition result from the last example, examine whether this shock an the effect satisfies the condition.
        \vspace{60pt}
        \item Based on part (2), solve for $Y$ and $Y^*$ at initial equilibrium using only $m_1$, $m_2$, $x_1$ and $\epsilon$.
        \vspace{60pt}
    \end{enumerate}
\end{example}

\begin{example}[DIFFICULT!!! Two-Country Model with Home Bias]
    Consider a world with two open economies, Home and Foreign, each described by:
    \[
    \begin{aligned}
    Y &= C + I + G - \frac{IM}{\epsilon} + X,\\
    C &= c_0 + c_1(Y-T), \qquad 0<c_1<1,\ c_0>0,\\
    I &= b_0 + b_1 Y, \qquad b_1>0,\\
    G &= T = G^* = T^* = 1,\\
    IM &= \lambda\, m_1 Y + m_2 \epsilon, \qquad 0<\lambda\le 1,\ 0<m_1<1,\ m_2>0,\\
    X &= \lambda^*\, x_1 \frac{Y^*}{\epsilon}, \qquad 0<\lambda^*\le 1,\ 0<x_1<1.
    \end{aligned}
    \]
    Here $\lambda$ is proportion of tradable goods, while for domestic output, $1-\lambda$ proportion are non-tradable goods. $\lambda^*$ is the tradable goods ratio for the foreign country.
    
    \begin{enumerate}[label=(\arabic*)]
    \item Use a system of two equations with unknowns $Y$ and $Y^*$ to characterize the world economy equilibrium output.
    \vspace{100pt}
    
    \item Holding $Y$ and $Y^*$ fixed, show that a depreciation ($\downarrow \epsilon$) improves the trade balance iff
    \[
    \lambda^* x_1 Y^*>\lambda m_1 Y.
    \]
    Rewrite this as a condition on the \emph{level} of net exports $NX:=X-IM/\epsilon$ and interpret the role of $\lambda,\lambda^*$ and $m_2$.
    \vspace{100pt}
    
    \item Take $(c_1,b_1)=(0.6,0.2)$, $c_0=0.2$, $b_0=0.1$, $m_1=0.3$, $m_2=0.1$, $x_1=0.4$, $\lambda=\lambda^*=0.6$, and $\epsilon=1$. 
    \begin{enumerate}[label=(\roman*)]
        \item Compute $(Y_{\text{eqm}},Y^*_{\text{eqm}})$. 
        \vspace{60pt}
        \item Compute $NX$ and verify whether the ML condition holds. 
        \vspace{60pt}
        \item Suppoe that $\epsilon'=0.8$ now. Compute the new equilibria output in the domestic and foreign economy. Does the result match your result in the previous part?
        \vspace{60pt}
        \item Suppose that $\lambda' = 0.5$ now. Compute the new equilibria output in the domestic and foreign economy. How to interpret the result? 
    \end{enumerate}
    \end{enumerate}
\end{example}
    
\end{document}