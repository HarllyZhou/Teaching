\documentclass[12pt]{article}

\usepackage[utf8]{inputenc}
\usepackage{geometry}
\geometry{a4paper,scale=0.75}
\linespread{1.5}
\usepackage{graphicx} 
\usepackage{float} 
\usepackage{subfig} 
\usepackage{enumerate}
\usepackage{enumitem}
\usepackage{amsmath}
\usepackage{array}
\usepackage{booktabs}
\usepackage{multirow}
\usepackage{amsfonts}
\usepackage[english]{babel}
\usepackage{amsthm}
\usepackage{dcolumn}
\usepackage{multicol}
\usepackage{stfloats}
\usepackage{lscape}
\usepackage[figuresright]{rotating}
\RequirePackage{pdflscape}
\usepackage[toc,page]{appendix}
\usepackage{geometry}
\usepackage{longtable}
\usepackage{comment}
\usepackage{xcolor}

% -------- enumerated sub-labels (a), (b), … --
\usepackage{enumitem}
\setlist[enumerate,1]{label=(\alph*),ref=\alph*}
% ---------------------------------------------

\usepackage{hyperref}
\hypersetup{hidelinks,
	colorlinks=true,
	allcolors=black,
	pdfstartview=Fit,
	breaklinks=true}
\usepackage{csquotes}
\usepackage{natbib}
\bibliographystyle{apalike}
\newtheorem{definition}{Definition}
\newtheorem{theorem}{Theorem}
\newtheorem{proposition}[theorem]{Proposition}
\newtheorem{lemma}[theorem]{Lemma}
\newtheorem{corollary}[theorem]{Corollary}
\newtheorem*{remark}{Remark}
\newtheorem{example}{Example}
\newtheorem{exercise}{Exercise}
\newtheorem{assumption}{Assumption}[section] % number within sections


\begin{document}

\thispagestyle{empty}

\begin{center}
    THE HONG KONG UNIVERSITY OF SCIENCE AND TECHNOLOGY\\
    {\large \textbf{ECON 3123 Midterm Exam}}\\
    Solutions and Grading Rubrics
\end{center}

\paragraph{Multiple Choice Questions} D D C D C (4 points each)

\paragraph{Question 6 (20 points)}
\begin{enumerate}[label=\alph*.]
	\item Goods demand is $Z=c+I+G$. At equilibrium, $Z=Y$. Therefore, the equilibrium output is
	\[Y^* = \frac{1}{1-c_1(1-t_1)-b_1}[(c_0-c_1t_0+b_0+G)-b_2 i].\]
	The equilibrium taxes is
	\[T^* = t_0 + \frac{t_1}{1-c_1(1-t_1)-b_1}[(c_0-c_1t_0+b_0+G)-b_2 i].\]

	Grading: 1 point for equilibrium condition, 2 points for $Y^*$, 2 points for $T^*$,

	\item When $b_0$ drops the equilibrium taxes drop. As a result, the balanced budget requires a drop in $G$. Since output is increasing in both $b_0$ and $G$, the drop in output will be reinforced.
	
	Grading: 5 points for drop in taxes, 5 points for reinforcement.

	\item Drops in $Y$ and $b_0$ lead to a decrease in $I$. Since $I = S+T-G$ and $G=T$, the private saving drops.
	
	Grading: 2 points for IS relation, 3 points for result.
\end{enumerate}

\paragraph{Question 7 (30 points)}
\begin{enumerate}[label=\alph*.]
	\item The goods demand is $Z=C+I+G$. At equilibrium, $Z=Y$. Therefore,
	\[Y = 2.75 - 5i.\]

	Grading: 2 points for equilibrium condition, 8 points for IS relation.

	\item At equilibrium, $H^s = H^d = H$ and $Y^* = 2.5$. Then
	\[H = [c + \theta(1-c)]M^d = [0.2 + 0.25\times(1-0.2)]\times 2\times 2.5 \times (0.7 - 4 \times 5\%) = 1.\]

	Grading: 2 points for equilibrium condition, 2 points for the formula, 6 points for solution.

	\item At equilibrium, $H^s = H^d = H'$ and $Y^* = 2.5$. Now $\theta' = 0.3$. 
	\[H = [c + \theta'(1-c)]M^d = [0.2 + 0.3\times(1-0.2)]\times 2\times 2.5 \times (0.7 - 4 \times 5\%) = 1.1.\]

	Grading: 1 point for new notation, 1 point for formula, 3 points for result.

	\item Recall that in equilibrium, $Y = 3 - 5(i+x)$. Since $x = 15\%$, $Y' = 2$ and $H^s = H^d = H'' = 0.88.$ 
	
	Grading: 1 point for $Y, i ,x$ relationship, 1 point for new notation, 1 point for formula, 2 points for result.
\end{enumerate}

\paragraph{Question 8 (30 point)}
\begin{enumerate}[label=\alph*.]
	\item $\$P_{2,t} = \frac{\text{Face Value}}{(1+i_{1,t+1}^e)(1+i_{1,t}+x)} = \$89.07.$
	\item $\$P_{3,t} = \frac{\text{Face Value}}{(1+i_{1,t+2}^e)(1+i_{1,t+1}^e+x)(1+i_{1,t}+x)} = \$83.28.$
	\item $y_{3.t} = \left(\frac{\text{Face Value}}{\$P_{3,t}}\right)^{\frac{1}{3}}=6.29\%.$
\end{enumerate}
Grading: 5 points for formula, 5 points for results.


\end{document}