\documentclass[12pt]{article}

\usepackage[utf8]{inputenc}
\usepackage{geometry}
\geometry{a4paper,scale=0.75}
\linespread{1.5}
\usepackage{graphicx} 
\usepackage{float} 
\usepackage{subfig} 
\usepackage{enumerate}
\usepackage{enumitem}
\usepackage{amsmath}
\usepackage{array}
\usepackage{booktabs}
\usepackage{multirow}
\usepackage{amsfonts}
\usepackage[english]{babel}
\usepackage{amsthm}
\usepackage{dcolumn}
\usepackage{multicol}
\usepackage{stfloats}
\usepackage{lscape}
\usepackage[figuresright]{rotating}
\RequirePackage{pdflscape}
\usepackage[toc,page]{appendix}
\usepackage{geometry}
\usepackage{longtable}
\usepackage{comment}
\usepackage{xcolor}

% -------- enumerated sub-labels (a), (b), … --
\usepackage{enumitem}
\setlist[enumerate,1]{label=(\alph*),ref=\alph*}
% ---------------------------------------------

\usepackage{hyperref}
\hypersetup{hidelinks,
	colorlinks=true,
	allcolors=black,
	pdfstartview=Fit,
	breaklinks=true}
\usepackage{csquotes}
\usepackage{natbib}
\bibliographystyle{apalike}
\newtheorem{definition}{Definition}
\newtheorem{theorem}{Theorem}
\newtheorem{proposition}[theorem]{Proposition}
\newtheorem{lemma}[theorem]{Lemma}
\newtheorem{corollary}[theorem]{Corollary}
\newtheorem*{remark}{Remark}
\newtheorem{example}{Example}
\newtheorem{exercise}{Exercise}
\numberwithin{equation}{section}
\newtheorem{assumption}{Assumption}[section] % number within sections

\begin{document}

\begin{center}
    ECON 3123: Macroeconomic Theory I\\
    {\large \textbf{Tutorial Note 2: Consumption and Goods Market}}\\
    Solution to Exercises\\
    Teaching Assistant: Harlly Zhou
\end{center}

\begin{enumerate}[label=\arabic*.]
    \item \begin{enumerate}[label=\alph*.]
        \item $Y=480+(0.5)(Y-70)+110+250=1610\text{billion}$.
        \item $Y_D = 1610 - 70 = 1540\text{billion}$.
        \item $C = 480 + 0.5(1540) = 1250 \text{billion}$.
        \end{enumerate}
    \item \begin{itemize}
        \item Q5(c): Because of the automatic effect of taxes on the economy, the economy responds less to changes in autonomous spending than in the case where taxes are independent of income. Since output tends to vary less (to be more stable), fiscal policy is called an automatic stabilizer.
        \item Q6(c): Both Y and T decrease.
        \item Q6(d): If $G$ is cut, $Y$ decreases even more. A balanced budget requirement amplifies the effect of the decline in $c_0$. Therefore, such a requirement is destabilizing.
        \end{itemize}
    \item A. Simple by definition.
    \item (1) Transfers will increase during recessions when output $Y$ decreases. For example, in recessions, the unemployment rate will increase. The government pays parts of workers' original earnings for a specified amount of time. This would help the unemployed workers and reduce the negative effects of recession on consumption.

    (2) a. $Y = \frac{1}{1-c_1(1-r_2)}[(c_0+c_1r_1) - c_1 T + I + G]$.

    b. $Y$ first increases by $\frac{1-c_1}{1-c_1(1-r_2)}$ by the equilibrium output. Then the consumption increases by $\frac{c_1(1-c_1)}{1-c_1(1-r_2)}$ via the behavioral equation. The equilibrium condition implies that $Y$ further increases by $\frac{c_1(1-c_1)}{1-c_1(1-r_2)}$. The iteration implies that the multiplier will be
    \[\frac{1-c_1}{1-c_1(1-r_2)} \sum_{i=0}^{+\infty} c_1^i = \frac{1}{1-c_1(1-r_2)}.\]
    
    (3) a. With constant $\bar{T}$, the disposable income is
    \[Y_D = Y - \bar{T} + (r_1 - r_2 Y) = (1-r_2)Y - (\bar{T} - r_1).\]
    In the alternative system, the dispoable income is
    \[Y_D = Y - (t_0 + t_1 Y) = (1 - t_1)Y - t_0.\]
    Equating the coefficients yields the correspondence:
    \[t_1 = r_2, \qquad \qquad t_0 = \bar{T} - r_1.\]
    
    b. $r_2>0$ and $t_1>0$ are both automatic stabilizers that flattens the slope of $Y_D$ w.r.t. $Y$. Increasing $t_0$ is the same as either increasing the fixed tax $\bar{T}$ or lower the transfer $r_1$.
\end{enumerate}

\end{document}