\documentclass[xcolor=dvipsnames, aspectratio=1610]{beamer}

\usepackage{etex}
\usepackage[utf8]{inputenc}
\usepackage{tgtermes}
\usefonttheme{default}
\renewcommand{\familydefault}{\rmdefault}
\usetheme{Madrid}
\usepackage{lmodern}

% --- Autumn + Ethereum palette (keep your structure/names) ---
\definecolor{autumnyellow}{HTML}{FFB000} % golden yellow (primary accent)
\definecolor{maplered}{HTML}{C1121F}   % maple red
\definecolor{chestnutbrown}{HTML}{8B5E3C}     % chestnutbrown
\definecolor{canvasbg}{HTML}{FAF4E6}    % warm paper background
\definecolor{graphite}{HTML}{3C3C3D}    % neutral text
\colorlet{autumndark}{autumnyellow!80!black}
\colorlet{canvasdark}{canvasbg!80!black}

% --- Beamer colors (same structure as yours) ---
\setbeamercolor{structure}{fg=autumnyellow}
\setbeamercolor{background canvas}{bg=canvasbg}
\setbeamercolor{frametitle}{bg=autumnyellow, fg=white}
\setbeamercolor{title}{bg=autumnyellow, fg=white}
\setbeamercolor{item}{fg=autumnyellow}
\setbeamercolor{section in toc}{fg=maplered}
\setbeamercolor{subsection in toc}{fg=autumnyellow}

% Optional: readable body text color
\setbeamercolor{normal text}{fg=graphite,bg=canvasbg}

\usepackage{tikz}
\usetikzlibrary{shapes.geometric, arrows.meta, positioning}
\usepackage{graphicx} % For \resizebox

% --- TikZ styles (same names; colors mapped to autumn/ETH) ---
\tikzset{
  block/.style = {rectangle, draw, fill=autumnyellow!40, text width=4cm, align=center, rounded corners, minimum height=2.0em},
  sideblock/.style = {rectangle, draw, fill=chestnutbrown!40, text width=4cm, align=center, rounded corners, minimum height=2.0em},
  decision/.style = {diamond, draw, fill=canvasbg!40, text width=5cm, align=center, aspect=2, inner sep=1pt},
  arrow/.style = {thick, ->, >=Stealth},
  mechanismblock/.style = {rectangle, draw, fill=maplered!40, text width=4cm, align=center, minimum height=2.5em},
  greydashedarrow/.style = {thick, dashed, ->, >=Stealth, draw=gray},
  blusharrow/.style = {thick, ->, >=Latex, draw=maplered},
  blank/.style = {rectangle, fill=maplered!20, text width=4cm, align=center, minimum height=2.0em}
}

%<---- do not use enumitem, does not work well with Beamer

\usepackage{amsmath, amssymb, amsfonts}
\usepackage{booktabs, array, dcolumn}
\usepackage{graphicx, subfig}
\usepackage{epstopdf}
\def\pdfshellescape{1}

% Icons/bullets
\setbeamertemplate{itemize items}[ball]
\setbeamertemplate{itemize subitem}[triangle]

% Theorems
\usepackage{amsthm}
\newtheorem{proposition}{Proposition}
\newtheorem*{quiz}{Quiz}
\newtheorem{claim}[proposition]{Claim}
\newtheorem{exercise}[proposition]{Exercise}
\newtheorem{remark}[proposition]{Remark}

% Links: use Ethereum blue
\definecolor{links}{HTML}{627EEA}
\hypersetup{colorlinks,linkcolor=,urlcolor=links}

\setbeamertemplate{navigation symbols}{}
\newcommand{\hilight}[1]{\colorbox{yellow}{#1}}

\DeclareMathOperator*{\plim}{plim}
\DeclareMathOperator*{\argmax}{arg\,max}
\DeclareMathOperator*{\E}{E}
\DeclareMathOperator*{\Var}{Var}
\DeclareMathOperator*{\Cov}{Cov}
\DeclareMathOperator*{\Corr}{Corr}
\DeclareMathOperator*{\supp}{supp}

\newcommand{\ind}{\mathrel{\perp \! \! \! \perp}}
\def\citeapos#1{\citeauthor{#1}'s (\citeyear{#1})}

\title[Measurment of Macroeconomy]{Tutorial 2: Consumption and Goods Market}
\subtitle{ECON 3123: Macroeconomic Theory I}
\author[Harlly Zhou]{Harlly Zhou}
\institute[HKUST]{Department of Economics\\
HKUST Business School}
\date{}

\begin{document}

%%%%%%%%%%%%%%%%%%%%%%%%%%%%%%%%%%%%%%%%%%%%
%%%%%%%%%%%%%%%%%%%%%%%%%%%%%%%%%%%%%%%%%%%%
\begin{frame}
\titlepage
\end{frame}

\begin{frame}{Example 1: Automatic Stabilizers}
    Chapter 3, Question 5 (a)(b) and Question 6 (b) in Blanchard, Olivier (2021), \textit{Macroeconomics}, 8th ed., Pearson.

    Consider the following behavioral equations:
    \begin{align*}
        C &= c_0 + c_1Y_D\\
        T &= t_0 + t_1Y\\
        Y_D &= Y-T
    \end{align*}
    where $G$ and $I$ are constants. Assume that $t_1\in(0,1)$.
    \begin{enumerate}
        \item Solve for the equilibrium output.
        \item What is the multiplier? Does the economy respond more to changes in autonomous spending when $t_1=0$ or $t_1>0$? Explain.
        \item Solve for taxes in equilibrium.
    \end{enumerate}
\end{frame}

\begin{frame}{Example 2: Chain RGDP}
    Consider an economy characterized by the following behavioral equations:
    \begin{align*}
        C &= c_0 + c_1 Y_D\\
        Y_D &= Y - T\\
        T &= t_1 Y + t_2 C
    \end{align*}
    where $t_1, t_2 \in (0,1)$. $G$ and $I$ are given. This is case when both income and consumption are taxed. The economy is now at its equilibrium.
    \begin{enumerate}
        \item Solve for the equilibrium output.
        \item What is the multiplier? Does this form of tax stabilizes output changes when there is a change in $c_0$, comparing with exogenous tax? Discuss cases where it does and it does not based on the equilibrium in part (1).
        \item Suppose that $c_0$ increases by 1 unit. In the new equilibrium, will consumption also increase by 1 unit? Discuss cases where it will and it will not based on the equilibrium in part (1).
        \item Write equilibrium saving as a function of $Y$.
        \item What is the MPS? Show that when $c_0$ increases by 1 unit, if $t_1+t_2=1$, the new equilibrium saving will decrease by 1 unit.
    \end{enumerate} 
\end{frame}


\end{document}