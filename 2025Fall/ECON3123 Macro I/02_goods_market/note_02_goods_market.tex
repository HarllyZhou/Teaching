\documentclass[12pt]{article}

\usepackage[utf8]{inputenc}
\usepackage{geometry}
\geometry{a4paper,scale=0.75}
\linespread{1.5}
\usepackage{graphicx} 
\usepackage{float} 
\usepackage{subfig} 
\usepackage{enumerate}
\usepackage{enumitem}
\usepackage{amsmath}
\usepackage{array}
\usepackage{booktabs}
\usepackage{multirow}
\usepackage{amsfonts}
\usepackage[english]{babel}
\usepackage{amsthm}
\usepackage{dcolumn}
\usepackage{multicol}
\usepackage{stfloats}
\usepackage{lscape}
\usepackage[figuresright]{rotating}
\RequirePackage{pdflscape}
\usepackage[toc,page]{appendix}
\usepackage{geometry}
\usepackage{longtable}
\usepackage{comment}
\usepackage{xcolor}

% -------- enumerated sub-labels (a), (b), … --
\usepackage{enumitem}
\setlist[enumerate,1]{label=(\alph*),ref=\alph*}
% ---------------------------------------------

\usepackage{hyperref}
\hypersetup{hidelinks,
	colorlinks=true,
	allcolors=black,
	pdfstartview=Fit,
	breaklinks=true}
\usepackage{csquotes}
\usepackage{natbib}
\bibliographystyle{apalike}
\newtheorem{definition}{Definition}
\newtheorem{theorem}{Theorem}
\newtheorem{proposition}[theorem]{Proposition}
\newtheorem{lemma}[theorem]{Lemma}
\newtheorem{corollary}[theorem]{Corollary}
\newtheorem*{remark}{Remark}
\newtheorem{example}{Example}
\newtheorem{exercise}{Exercise}
\newtheorem{assumption}{Assumption}[section] % number within sections


\begin{document}

\begin{center}
    ECON 3123: Macroeconomic Theory I\\
    {\large \textbf{Tutorial Note 2: Consumption and Goods Market}}\\
    Teaching Assistant: Harlly Zhou
\end{center}



\subsection*{Consumption and Keynesian Cross}
\paragraph{Consumption Function}
The main factor that determines consumption is \textbf{disposable income}, denoted by $Y_D$. It is the income that remains once consumers receive transfers from the government and pay their taxes:
\begin{align*}
    Y_D = Y - T.
\end{align*}

We assume that the consumption satisfies the following linear relation:
\begin{align*}
    C = c_0 + c_1 Y_D = c_0 + c_1 (Y-T).
\end{align*}
This is a behavioral equation. 
\begin{enumerate}
    \item The parameter $c_0$, autonomous consumption, captures the consumption when $Y_D=0$: subsistence level of consumption, and effects of other factors.
    \item The parameter $c_1$, marginal propensity to consume (MPC),
    captures the effect an additional dollar of disposable income has on consumption.
\end{enumerate}

\paragraph{Keynesian Cross}
Assume that investment value is exogenously given as
\begin{align*}
    I = \bar{I},
\end{align*}
and that $NX=0$. 
The demand for goods is
\begin{align}\label{eq:demand_v1}
    Z &\equiv C + I + G + NX \notag \\
    &= C + \bar{I} + G \notag \\
    &= [c_0 + c_1(Y-T)] + \bar{I} + G \notag \\
    &= (c_0 + \bar{I} + G - c_1T) + c_1Y.
\end{align}
Given \textbf{income} $Y$, people want to purchase $Z$ amount of goods and services.

The supply for goods is the total production $Y$. 

The equilibrium condition is
\begin{align}\label{eq:eqm_cond}
    \text{Demand } = \text{ Supply }\,\, \iff \,\, Z = Y.
\end{align}

\begin{figure}[htp]
    \centering
    \includegraphics[width=0.6\textwidth]{keynesian_cross_0.png}
    \caption{Goods Market Equilibrium, Keynesian Cross}
    \label{fig:key_cross_v1}
\end{figure}

Figure \ref{fig:key_cross_v1} graphically shows the equilibrium. 
\begin{itemize}
    \item On the supply side, given income $Y$, we always have income equal to production. So it is the blue 45 degree line.
    \item On the demand side, we assume that $c_0 + \bar{I} + G - c_1T > 0$ and $c_1 > 0$. Since we typically have $c_1<1$ (why?), this ensures the existence of equilibrium.
\end{itemize}

\paragraph{Autonomous Spending and Multiplier}
Now consider increasing the autonomous consumption. This moves the demand line upward so that the equilibrium income and expenditure both increase. This is shown in Figure \ref{fig:key_cross_v2}.

\begin{figure}[htp]
    \centering
    \includegraphics[width=0.6\textwidth]{keynesian_cross_c0change.png}
    \caption{Increasing $c_0$ moves demand upward}
    \label{fig:key_cross_v2}
\end{figure}

We would like to know how equilibrium output change from point $A$ to point $D$ when $c_0$ increases to $c_0'$. This idea is captured by the concept of \textbf{multiplier}. The multiplier implies how much output will increase given a unit increase in autonomous spending. 

Graphically, we can decompose the increase from $A$ to $D$ into multiple rounds. In the $n$-th round of increase, the output increases by $c_1^{n-1}$ unit. Summing up all the rounds, we get a geometric series:
\begin{align*}
    1 + c_1 + c_1^2 + \cdots + c_1^n + \cdots = \sum_{i=1}^{+\infty} c_1^{i-1} = \frac{1}{1-c_1}.
\end{align*}

Algebraically, substituting the demand identity \eqref{eq:demand_v1} into equilibrium condition \eqref{eq:eqm_cond}, we get
\begin{align*}
    Y = (c_0 + \bar{I} + G - c_1T) + c_1Y.
\end{align*}
This is equivalent to
\begin{align*}
    Y = \frac{1}{1-c_1} (c_0 + \bar{I} + G - c_1T).
\end{align*}
Holding other variables constant, if we increase $c_0$ by 1 unit, then $Y$ increases by $\frac{1}{1-c_1}$ units.

\paragraph{MPC and Multiplier}
Figure \ref{fig:key_cross_v3} illustrates the change of equilibrium with two different demand lines that differ only in $c_1$, the marginal propensity to consumption. We notice that given a same amount of increase in the autonomous spending, the increase of equilibrium output is larger for demand line 2 whose MPC is larger.

\begin{figure}[htp]
    \centering
    \includegraphics[width=0.6\textwidth]{keynesian_cross_c1change.png}
    \caption{Increasing $c_1$ yields larger multiplier}
    \label{fig:key_cross_v3}
\end{figure}

Algebraically, when $c_1$ increases, the multiplier $\frac{1}{1-c_1}$ also increases.

\begin{exercise}
    Chapter 3, Question 2 in Blanchard, Olivier (2021), \textit{Macroeconomics}, 8th ed., Pearson.
\end{exercise}

\begin{example}
    Chapter 3, Question 5 (a)(b) and Question 6 (b) in Blanchard, Olivier (2021), \textit{Macroeconomics}, 8th ed., Pearson.

    Consider the following behavioral equations:
    \begin{align*}
        C &= c_0 + c_1Y_D\\
        T &= t_0 + t_1Y\\
        Y_D &= Y-T
    \end{align*}
    where $G$ and $I$ are constants. Assume that $t_1\in(0,1)$.
    \begin{enumerate}[label=\alph*.]
        \item Solve for the equilibrium output.
        \item What is the multiplier? Does the economy respond more to changes in autonomous spending when $t_1=0$ or $t_1>0$? Explain.
        \item Solve for taxes in equilibrium.
    \end{enumerate}
\end{example}

\vspace{36pt}

\begin{exercise}
    Chapter 3, Question 5 (c) and Question 6 (c)(d) in Blanchard, Olivier (2021), \textit{Macroeconomics}, 8th ed., Pearson.
\end{exercise}

\subsection*{Savings}
\paragraph{Private Saving and Public Saving}
\textbf{Private saving} equals disposable income minus consumption:
\begin{align}\label{eq:def_priv_s}
    S = Y_D - C = Y - T - C.
\end{align}
\textbf{Public saving} equals taxes (net of transfers) minus government spending:
\begin{align*}
    T - G
\end{align*}

\paragraph{Goods Market Equilibrium and IS relation}
By \eqref{eq:demand_v1} and \eqref{eq:eqm_cond}, the equilibrium condition can be rewritten as
\begin{align}\label{eq:eqm_cond_v2}
    Y = C + I + G.
\end{align}
Rewriting the condition, we get
\begin{align*}
    Y - T - C = I + G - T.
\end{align*}
Note that that left-hand side (LHS) of the equation is private savings $S$. Rearranging the terms, we get the IS relation:
\begin{align*}
    I = S + (T-G),
\end{align*}
\textbf{I}nvestment equals \textbf{S}avings. More specifically, IS relation implies that at goods market equilibrium, the amount that firms want to invest must equal the amount that people and the government want to save.

As we have just shown, we can alternatively think about goods-market equilibrium as the condition that investment equals savings.

\paragraph{The Paradox of Saving}
Substituting the consumption function into \eqref{eq:def_priv_s}, we obtain
\begin{align*}
    S = -c_0 + (1-c_1)(Y-T).
\end{align*}
$1-c_1$ is called the \textbf{marginal propensity to save} (MPS).

What happens if we decrease saving by increasing $c_1$? In class, we showed that in equilibrium, since $I=\bar{I}$, $T=\bar{T}$ and $G=\bar{G}$, the IS relation implies that $S$ should be constant. Mathematically, we know that
\[Y_D = \frac{1}{1-c_1} (c_0 + I + G -T).\]
Then 
\begin{align*}
    S &= -c_0 + (1-c_1)(Y-T)\\
    &= I + G - T.
\end{align*}
We have the IS relation again and the argument repeats.

What happens if we decrease saving by increasing $c_0$? The argument should be similar to the changes in $c_1$. Do this by yourself as an exercise, both via the IS relation and by using math.

\begin{exercise}
    When a person gets an increase in current income, what is likely to happen to consumption and saving?
    \begin{enumerate}[label=\Alph*.]
        \item Consumption increases and saving increases.
        \item Consumption increases and saving decreases.
        \item Consumption decreases and saving increases.
        \item Consumption decreases and saving decreases.
    \end{enumerate}
\end{exercise}

\begin{example}
    Consider an economy characterized by the following behavioral equations:
    \begin{align*}
        C &= c_0 + c_1 Y_D\\
        Y_D &= Y - T\\
        T &= t_1 Y + t_2 C
    \end{align*}
    where $t_1, t_2 \in (0,1)$. $G$ and $I$ are given. This is case when both income and consumption are taxed. The economy is now at its equilibrium.
    \begin{enumerate}[label=(\arabic*)]
        \item Solve for the equilibrium output.
        \item What is the multiplier? Does this form of tax stabilizes output changes when there is a change in $c_0$, comparing with exogenous tax? Discuss cases where it does and it does not based on the equilibrium in part (1).
        \item Suppose that $c_0$ increases by 1 unit. In the new equilibrium, will consumption also increase by 1 unit? Discuss cases where it will and it will not based on the equilibrium in part (1).
        \item Write equilibrium saving as a function of $Y$.
        \item What is the MPS? Show that when $c_0$ increases by 1 unit, if $t_1+t_2=1$, the new equilibrium saving will decrease by 1 unit.
    \end{enumerate} 
\end{example}

\subsection*{Government and Fiscal Policy: Financial Stimulus}
Recall that at equilibrium, we have
\begin{align}
    Y &= Z = C + I + G \label{eq:eqm_fina_sti} \\
    &= \frac{1}{1-c_1}(c_0 + \bar{I} + G - c_1T),\label{eq:y_fina_sti}
\end{align}
where consumption satisfies the following behavioral equation:
\begin{align}\label{eq:c_fina_sti}
    C = c_0 + c_1(Y-T).
\end{align}
In the previous two examples, we have seen some ways to stabilize business cycles. Government can actually use financial stimulus to stabilize business cycle. Consider the following three alternatives:
\begin{enumerate}[label=(\arabic*)]
    \item Increase government spending by 1 unit while holding other accounts constant.
    
    If $G$ increases by 1 unit, then $Y$ increases by 1 unit via \eqref{eq:eqm_fina_sti}, so $C$ increases by $c_1$ unit via \eqref{eq:c_fina_sti}. This takes effect again in the equilibrium condition, which lets $Y$ increase by $c_1$ more unit via \eqref{eq:eqm_fina_sti}, thus $C$ increase by $c_1 \times c_1 = c_1^2$ unit via \eqref{eq:c_fina_sti}. Continued with the process, we get the \textbf{spending multiplier}:
    \begin{align*}
        \frac{\Delta Y}{\Delta G} = \frac{\sum_{i=0}^{+\infty}c_1^i}{1} = \frac{1}{1-c_1}.
    \end{align*}

    \item Decrease tax by 1 unit while holding other accounts constant.
    
    If $T$ decreases by 1 unit, then $C$ increases by $c_1$ unit via \eqref{eq:c_fina_sti}, so that $Y$ increases by $c_1$ unit via \eqref{eq:eqm_fina_sti}. This takes effect on $C$ which lets $C$ increase by $c_1\times c_1 = c_1^2$ unit via \eqref{eq:c_fina_sti} thus increasing $Y$ by $c_1^2$ more unit via \eqref{eq:eqm_fina_sti}. Continuing with this process, we get the \textbf{tax multiplier}:
    \begin{align*}
        \frac{\Delta Y}{\Delta T} = \frac{\sum_{{\color{red} i=1}}^{+\infty}c_1^i}{-1} = -\frac{c_1}{1-c_1}.
    \end{align*}

    \item Increase both government spending and tax by 1 unit while holding other accounts constant.
    
    This will give you the \textbf{balanced budget multiplier}. Try to derive this by yourself as a review of the class note.
\end{enumerate}

\begin{exercise}[Difficult]
    We now extend the model by introducing \textbf{transfers}. Suppose the economy is characterized by the following behavioral equations:
    \begin{align*}
        C &= c_0 + c_1 Y_D\\
        Y_D &= Y - T + R\\
        R &= r_1 - r_2 Y.
    \end{align*}
    $G$, $T$ and $I$ are constants. $r_1, r_2 \in (0,1)$.
    \begin{enumerate}[label=(\arabic*)]
        \item Explain the economic meaning of the negative coefficient of $Y$ in the transfer equation. (Hint: think about how transfers such as unemployment benefits change when output/income rises or falls.)
        \item \begin{enumerate}[label=\alph*.]
            \item Solve for the equilibrium $Y$.
            \item Consider a balanced budget where $G$ and $T$ increases by 1 unit at the same time. What is the balanced budget multiplier? 
        \end{enumerate}
        \item Consider the following system:
        \begin{align*}
            C &= c_0 + c_1 Y_D\\
            Y_D &= Y - T\\
            T &= t_0 + t_1 Y.
        \end{align*}
        \begin{enumerate}[label=\alph*.]
            \item Show how this system can be seen as an alternative representation of fiscal policy with automatic stabilizers.
            \item Explain the correspondence between parameters $(t_0, t_1)$ and $(r_1, r_2)$.
        \end{enumerate}
    \end{enumerate}
\end{exercise}
\end{document}