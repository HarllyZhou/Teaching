\documentclass[12pt]{article}

\usepackage[utf8]{inputenc}
\usepackage{geometry}
\geometry{a4paper,scale=0.75}
\linespread{1.5}
\usepackage{graphicx} 
\usepackage{float} 
\usepackage{subfig} 
\usepackage{enumerate}
\usepackage{enumitem}
\usepackage{amsmath}
\usepackage{array}
\usepackage{booktabs}
\usepackage{multirow}
\usepackage{amsfonts}
\usepackage[english]{babel}
\usepackage{amsthm}
\usepackage{dcolumn}
\usepackage{multicol}
\usepackage{stfloats}
\usepackage{lscape}
\usepackage[figuresright]{rotating}
\RequirePackage{pdflscape}
\usepackage[toc,page]{appendix}
\usepackage{geometry}
\usepackage{longtable}
\usepackage{comment}
\usepackage{xcolor}

% -------- enumerated sub-labels (a), (b), … --
\usepackage{enumitem}
\setlist[enumerate,1]{label=(\alph*),ref=\alph*}
% ---------------------------------------------

\usepackage{hyperref}
\hypersetup{hidelinks,
	colorlinks=true,
	allcolors=black,
	pdfstartview=Fit,
	breaklinks=true}
\usepackage{csquotes}
\usepackage{natbib}
\bibliographystyle{apalike}
\newtheorem{definition}{Definition}
\newtheorem{theorem}{Theorem}
\newtheorem{proposition}[theorem]{Proposition}
\newtheorem{lemma}[theorem]{Lemma}
\newtheorem{corollary}[theorem]{Corollary}
\newtheorem*{remark}{Remark}
\newtheorem{example}{Example}
\newtheorem{exercise}{Exercise}
\numberwithin{equation}{section}
\newtheorem{assumption}{Assumption}[section] % number within sections


\begin{document}

\begin{center}
    ECON 3123: Macroeconomic Theory I\\
    {\large \textbf{Tutorial Note 2: Consumption and Goods Market}}\\
    Teaching Assistant: Harlly Zhou
\end{center}

\subsection*{Decomposition of GDP}
GDP is the sum of consumption, investment, government spending, net export, and inverntory investment, which we always ignored due to its relatively tiny size.
\begin{align*}
    Y = C + I + G + NX,\,\,\,\, \text{where } NX = EX - IM.
\end{align*}
\begin{enumerate}[label=(\arabic*)]
    \item Consumption ($C$) is the purchase of goods and services by consumers. It is the largest component of GDP.
    \item Investment ($I$) is the sum of nonresidential investment (\textit{e.g.,} a new machine bought by firm) and residential investment (\textit{e.g., }purchase of a new house). 
    \item Government sepnding ($G$) is the purchase of goods and services by different layers of government. Note that government transfer is not government spending.
    \item Exports ($EX$) are purchase of domestic goods by foreigners. Imports ($IM$) are purvchases of foreign goods by domestic consumers, firms and government. Net exports ($NX$) is the difference between exports and imports. It can be negative. 
    \item Invertory investment is the difference betwee nproduction and purchases. It can be negative.
\end{enumerate}

\begin{exercise}
    \begin{enumerate}[label=(\arabic*)]
        \item Which of the following is not a category of consumption spending in the national income accounts?
        \begin{enumerate}[label=\Alph*.]
            \item Consumer durables
            \item Nondurable goods
            \item Services
            \item Housing purchases
        \end{enumerate}
        \item In the expenditure approach to GDP, which of the following would be excluded from measurements of GDP?
        \begin{enumerate}[label=\Alph*.]
            \item Government payments for goods produced by foreign firms
            \item Government payments for goods produced by firms owned by state or local governments
            \item Government payments for welfare 
            \item All government payments are included in GDP.Housing purchases
        \end{enumerate}
    \end{enumerate}
\end{exercise}

\subsection*{Consumption and Keynesian Cross}
\paragraph{Consumption Function}
The main factor that determines consumption is \textbf{disposable income}, denoted by $Y_D$. It is the income that remains once consumers receive transfers from the government and pay their taxes:
\begin{align*}
    Y_D = Y - T.
\end{align*}

We assume that the consumption satisfies the following linear relation:
\begin{align*}
    C = c_0 + c_1 Y_D = c_0 + c_1 (Y-T).
\end{align*}
This is a behvioral equation. 
\begin{enumerate}
    \item The parameter $c_0$, autonomous consumption, captures the consumption when $Y_D=0$: subsistence level of consumption, and effects of other factors.
    \item The parameter $c_1$, marginal propensity to consume (MPC),
    captures the effect an additional dollar of disposable income has on consumption.
\end{enumerate}

\paragraph{Keynesian Cross}
Assume that investment value is exogenously given as
\begin{align*}
    I = \bar{I},
\end{align*}
and that $NX=0$. 
The demand for goods is
\begin{align}\label{eq:demand_v1}
    Z &\equiv C + I + G + NX\\
    &= C + \bar{I} + G\\
    &= [c_0 + c_1(Y-T)] + \bar{I} + G\\
    &= (c_0 + \bar{I} + G - c_1T) + c_1Y.
\end{align}
Given \textbf{income} $Y$, people want to purchase $Z$ amount of goods and services.

The supply for goods is the total production $Y$. 

The equilibrium condition is
\begin{align}\label{eq:eqm_cond}
    \text{Demand } = \text{ Supply }\,\, \iff \,\, Z = Y.
\end{align}

\begin{figure}[htp]
    \centering
    \includegraphics[width=0.8\textwidth]{keynesian_cross_0.png}
    \caption{Goods Market Equilibrium, Keynesian Cross}
    \label{fig:key_cross_v1}
\end{figure}

Figure \ref{fig:key_cross_v1} graphically shows the equilibrium. 
\begin{itemize}
    \item On the supply side, given income $Y$, we always have income equal to production. So it is the blue 45 degree line.
    \item On the demand side, we assume that $c_0 + \bar{I} + G - c_1T > 0$ and $c_1 > 0$. Since we typically have $c_1<1$ (why?), this ensures the existence of equilibrium.
\end{itemize}

\paragraph{Autonomous Spending and Multiplier}
Now consider increasing the autonomous consumption. This moves the demand line upward so that the equilibrium income and expenditure both increase. This is shown in Figure \ref{fig:key_cross_v2}.

\begin{figure}[htp]
    \centering
    \includegraphics[width=0.8\textwidth]{keynesian_cross_c0change.png}
    \caption{Increasing $c_0$ moves demand upward}
    \label{fig:key_cross_v2}
\end{figure}

We would like to know how equilibrium output change from point $A$ to point $D$ when $c_0$ increases to $c_0'$. This idea is captured by the concept of \textbf{multiplier}. The multiplier implies how much output will increase given a unit increase in autonomous spending. 

Graphically, we can decompose the increase from $A$ to $D$ into multiple rounds. In the $n$-th round of increase, the output increases by $c_1^{n-1}$ unit. Summing up all the rounds, we get a geometric series:
\begin{align*}
    1 + c_1 + c_1^2 + \cdots + c_1^n + \cdots = \sum_{i=1}^{+\infty} c_1^{i-1} = \frac{1}{1-c_1}.
\end{align*}

Algebraically, substituting \eqref{eq:demand_v1} into \eqref{eq:eqm_cond}, we get
\begin{align*}
    Y = (c_0 + \bar{I} + G - c_1T) + c_1Y.
\end{align*}
This isequivalent to
\begin{align*}
    Y = \frac{1}{1-c_1} (c_0 + \bar{I} + G - c_1T).
\end{align*}
Holding other variables constant, if we inrease $c_0$ by 1 unit, then $Y$ increases by $\frac{1}{1-c_1}$ units.

\paragraph{MPC and Multiplier}
Figure \ref{fig:key_cross_v3} illustrates the change of equilibrium with two different demand lines that differ only in $c_1$, the marginal propensity to consumption. We notice that given a same amount of increase in the autonomous spending, the increase of equilibrium output is larger for demand line 2 whose MPC is larger.

\begin{figure}[htp]
    \centering
    \includegraphics[width=0.8\textwidth]{keynesian_cross_c1change.png}
    \caption{Increasing $c_1$ yields larger multiplier}
    \label{fig:key_cross_v3}
\end{figure}

Algebraically, when $c_1$ increases, $\frac{1}{1-c_1}$ also increases.

\begin{exercise}
    Chapter 3, Question 2 in Blanchard, Olivier (2021), \textit{Macroeconomics}, 8th ed., Pearson.
\end{exercise}

\begin{example}
    Chapter 3, Question 5 (a)(b) and Question 6 (b) in Blanchard, Olivier (2021), \textit{Macroeconomics}, 8th ed., Pearson.

    [Words omitted.] Consider the following behavioral equations:
    \begin{align*}
        C &= c_0 + c_1Y_D\\
        T &= t_0 + t_1Y\\
        Y_D &= Y-T
    \end{align*}
    where $G$ and $I$ are constants. Assume that $t_1\in(0,1)$.
    \begin{enumerate}[label=\alph*.]
        \item Solve for the equilibrium output.
        \item What is the multiplier? Does the economy respond more to changes in autonomous spending when $t_1=0$ or $t_1>0$? Explain.
        \item Solve for taxes in equilibrium.
    \end{enumerate}
\end{example}

\vspace{36pt}

\begin{exercise}
    Chapter 3, Question 5 (c) and Question 6 (c)(d) in Blanchard, Olivier (2021), \textit{Macroeconomics}, 8th ed., Pearson.
\end{exercise}


\end{document}