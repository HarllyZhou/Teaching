\documentclass[12pt]{article}

\usepackage[utf8]{inputenc}
\usepackage{geometry}
\geometry{a4paper,scale=0.75}
\linespread{1.5}
\usepackage{graphicx} 
\usepackage{float} 
\usepackage{subfig} 
\usepackage{enumerate}
\usepackage{enumitem}
\usepackage{amsmath}
\usepackage{array}
\usepackage{booktabs}
\usepackage{multirow}
\usepackage{amsfonts}
\usepackage[english]{babel}
\usepackage{amsthm}
\usepackage{dcolumn}
\usepackage{multicol}
\usepackage{stfloats}
\usepackage{lscape}
\usepackage[figuresright]{rotating}
\RequirePackage{pdflscape}
\usepackage[toc,page]{appendix}
\usepackage{geometry}
\usepackage{longtable}
\usepackage{comment}
\usepackage{xcolor}

% -------- enumerated sub-labels (a), (b), … --
\usepackage{enumitem}
\setlist[enumerate,1]{label=(\alph*),ref=\alph*}
% ---------------------------------------------

\usepackage{hyperref}
\hypersetup{hidelinks,
	colorlinks=true,
	allcolors=black,
	pdfstartview=Fit,
	breaklinks=true}
\usepackage{csquotes}
\usepackage{natbib}
\bibliographystyle{apalike}
\newtheorem{definition}{Definition}
\newtheorem{theorem}{Theorem}
\newtheorem{proposition}[theorem]{Proposition}
\newtheorem{lemma}[theorem]{Lemma}
\newtheorem{corollary}[theorem]{Corollary}
\newtheorem*{remark}{Remark}
\newtheorem{example}{Example}
\newtheorem{exercise}{Exercise}
\newtheorem{assumption}{Assumption}[section] % number within sections


\begin{document}

\begin{center}
    ECON 3123: Macroeconomic Theory I\\
    {\large \textbf{Tutorial Note 4: Investment and Financial Market}}\\
    Solution to Exercises\\
    Teaching Assistant: Harlly Zhou
\end{center}

\begin{enumerate}[label=\arabic*.]
    \item (a) $Y = 840 - 2000 i$.
    
    (b) $Y = 780$.

    (c) $M/P = 1440$.

    (d) $C=304$; $I=276$; $G=200$; $C+I+G=780$.

    (e) $Y = 795$; $C =308.5$; $I = 286.5$. The increase in the money supply decreases the interest rate. Consumption and investment increase because output increases and interest rates decrease.

    (f) At the initial rate of 3\%, $Y$ equals 980 when $G$ is increased to 300. A fiscal expansion increases output. Consumption increases ($C = 364$) because output increases. When the central bank keeps interest rates at 3\% then investment increases ($I = 316$) as output increases.
    \item (1) B. (2) D. (3) B.
\end{enumerate}






\end{document}