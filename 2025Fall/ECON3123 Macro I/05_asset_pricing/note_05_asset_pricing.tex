\documentclass[12pt]{article}

\usepackage[utf8]{inputenc}
\usepackage{geometry}
\geometry{a4paper,scale=0.75}
\linespread{1.5}
\usepackage{graphicx} 
\usepackage{float} 
\usepackage{subfig} 
\usepackage{enumerate}
\usepackage{enumitem}
\usepackage{amsmath}
\usepackage{array}
\usepackage{booktabs}
\usepackage{multirow}
\usepackage{amsfonts}
\usepackage[english]{babel}
\usepackage{amsthm}
\usepackage{dcolumn}
\usepackage{multicol}
\usepackage{stfloats}
\usepackage{lscape}
\usepackage[figuresright]{rotating}
\RequirePackage{pdflscape}
\usepackage[toc,page]{appendix}
\usepackage{geometry}
\usepackage{longtable}
\usepackage{comment}
\usepackage{xcolor}

% -------- enumerated sub-labels (a), (b), … --
\usepackage{enumitem}
\setlist[enumerate,1]{label=(\alph*),ref=\alph*}
% ---------------------------------------------

\usepackage{hyperref}
\hypersetup{hidelinks,
	colorlinks=true,
	allcolors=black,
	pdfstartview=Fit,
	breaklinks=true}
\usepackage{csquotes}
\usepackage{natbib}
\bibliographystyle{apalike}
\newtheorem{definition}{Definition}
\newtheorem{theorem}{Theorem}
\newtheorem{proposition}[theorem]{Proposition}
\newtheorem{lemma}[theorem]{Lemma}
\newtheorem{corollary}[theorem]{Corollary}
\newtheorem*{remark}{Remark}
\newtheorem{example}{Example}
\newtheorem{exercise}{Exercise}
\newtheorem{assumption}{Assumption}[section] % number within sections


\begin{document}

\begin{center}
    ECON 3123: Macroeconomic Theory I\\
    {\large \textbf{Tutorial Note 5: Asset Pricing}}\\
    Teaching Assistant: Harlly Zhou
\end{center}

\subsection*{Discounted Cash Flow}
The key concept is the \textbf{time value of money}. In words, it says that one dollar today does not equal one dollar tomorrow. Based on the interest rate in each time period, we can \textbf{discount} the future cash flow into \textbf{present value}. Why we need to do this? Suppose we have a 2-year bond with 2\% coupon and a 3 year bond with 2.1\% coupon. If we would like to compare the price of the two bonds, it is not clear which one has a higher price. Therefore, discounting allows us to compare values of assets that have different rate, time to maturity, and other factors influencing cash flow.

Suppose that in year $t+j$, the cash flow is $\$z_{t+j}$. Then the value of the asset with maturity $t+\tau$ at $t$ will be
\[\$V_t = \$ z_t + \frac{1}{1+i_t}\$ z_{t+1} + \frac{1}{(1+i_t)(1+i_{t+1})}\$ z_{t+2} + \cdots.\]
This complicated formula simply says that the present value of an asset is the sum of total \textbf{discount cash flow} in the future.

Now we restrict our discussion on constant interest rate $i$ and same payment each period $\$z$.

\paragraph{Annuity and Perpetuity} \textbf{Annuity} is a contract or financial product that provides a stream of periodic payments over time. \textbf{Perpetuity} is a special type of annuity that is \textit{without maturity date}, that is, makes payments forever. The formula of pricing an annuity with maturity $t+\tau$ is
\[\$ V_t = \$z \sum_{j=0}^{\tau-1} \frac{1}{(1+i)^j} = \$z \frac{1-\frac{1}{(1+i)^\tau}}{1-\frac{1}{1+i}}.\]
To price a perpetuity, we take $\tau \rightarrow +\infty$, and we have
\[\$ V_t = \frac{\$z}{i}\]

When there is uncertainty, we make everything in expectation. When the payments are real, we discount using real interest rates.

\begin{exercise}
    Suppose there is an asset that pays you \$1,000 at the end of each year from 2026 to 2030. The one year nominal interest rate is 3\% from 2025 to 2026, while it is expected that the nominal interest rate will be reduced to 2\% from the end of 2026 (2026 to 2027 and onwards). What will be a no-arbitrage price of the asset at the end of 2025?
\end{exercise}

\subsection*{Bond Pricing and Bond Yield}
Given a one-year zero coupon risk-free bond and a two-year zero coupon risk-free bond, no-arbitrage condition implies that
\[\$P_{2,t} (1 + i_{1,t}) = \$P_{1, t+1}^e,\]
meaning that the value of the 2-year bond after year should equal the value of a 1-year bond in the next year. Then we have the following relation:
\[(1 + i_{2,t})^2 = (1 + i_{1,t})(1 + i_{1, t+1}^e),\]
which can be approximated by
\[ i_{2,t} \approx \frac{1}{2}(i_{1,t} + i_{1, t+1}^e).\]

If we consider the default risk, then we need to add risk premium to the two-year rate:
\[ i_{2,t} \approx \frac{1}{2}(i_{1,t} + i_{1, t+1}^e + x).\]

The bond pricing formula is
\[ P_{\tau,t} = \frac{\text{Face Value}}{(1+i_{1,t})\prod_{j = 1}^{\tau-1}(1+i_{1,t+j}^e+x)}.\]

\begin{exercise}
    Derive the formula for the bond yield of a three-year zero-coupon risk-free bond using only one-year zero-coupon risk-free bond.
\end{exercise}

\begin{exercise}
    True or false: A prediction of recession is consistent with an upward-sloping yield curve.
\end{exercise}

What if you are pricing a bond with coupon given the annual yield? You add up two streams of future cash flows: the periodic coupon and the lump-sum face value in the end.

\subsection*{Stock Pricing}
Instead of the no-arbitrage argument we had in class, a simpler way is to look at stock pricing from the perspective of discounted cash flow. Since firms pay dividends to the investors infinitely, we simply add up the discounted values of dividends each period:
\[ \$Q_t = \$D_t + \sum_{j=1}^{+\infty} \$D_{t+j} \frac{1}{\prod_{k=1}^j(1+r_{t+k} + x)}. \]
With constant rates and constant dividend, we have
\[ \$Q_t = \frac{\$D}{r+x}. \]

Theoretically, as long as there is no new information in the market, the stock price should be constant. However, stock prices always fluctuate in reality. 

\begin{exercise}
    Chapter 14, Question 6 in Blanchard, Olivier (2021), \textit{Macroeconomics}, 8th ed., Pearson.
\end{exercise}

\begin{example}
    Chapter 14, Question 7 in Blanchard, Olivier (2021), \textit{Macroeconomics}, 8th ed., Pearson.

    Suppose that an investor has a choice between buying a 3-year bond with a face value of \$60 and a stock paying a constant dividend of \$20 per year, which the investor plans to hold for three years. The real interest rate on the stock and the bond is the same, 5\%. in addition, the risk premium on the stock is constant at 10\%, while on the bond, it is 5\%.
    \begin{enumerate}[label=(\arabic*)]
        \item Compare the present values of the two investments. Which one should the investor choose?
        \vspace{60pt}
        \item Imagine that the risk premia on the stock and the bond were to be equalized at 5\%. How would that affect the choice made in part (1)?
        \vspace{60pt}
        \item Suppose now that risk premium returns to its initial value, which is 10\% for the stock and 5\% for the bond. The interest rate changes each year for bothsecurities: 5\% in the first year, 8\% in the second, and 12\% in the third. How would that affect the investor's valuation of the two investments? How would you explain your result?
        \vspace{60pt}
    \end{enumerate}
\end{example}

\subsection*{Risks, Bubbles, and the Asset Market}
\paragraph{Risks} People's perception of risk changes across periods. In reality, we do not have a constant $x$ but $x_{t+j}$ changing with time. We saw this in the Argentina example in the previous note.

\paragraph{Mispricing} Mispricing can happen in different scenarios. Two of examples are rational speculative bubbles and irrational expectations.

\begin{exercise}
    Chapter 14, Question 8 (a) to (d) in Blanchard, Olivier (2021), \textit{Macroeconomics}, 8th ed., Pearson. Note that (e) is a data question. You can do it if you are interested.
\end{exercise}
\end{document}