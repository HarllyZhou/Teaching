\documentclass[12pt]{article}

\usepackage[utf8]{inputenc}
\usepackage{geometry}
\geometry{a4paper,scale=0.75}
\linespread{1.5}
\usepackage{graphicx} 
\usepackage{float} 
\usepackage{subfig} 
\usepackage{enumerate}
\usepackage{enumitem}
\usepackage{amsmath}
\usepackage{array}
\usepackage{booktabs}
\usepackage{multirow}
\usepackage{amsfonts}
\usepackage[english]{babel}
\usepackage{amsthm}
\usepackage{dcolumn}
\usepackage{multicol}
\usepackage{stfloats}
\usepackage{lscape}
\usepackage[figuresright]{rotating}
\RequirePackage{pdflscape}
\usepackage[toc,page]{appendix}
\usepackage{geometry}
\usepackage{longtable}
\usepackage{comment}
\usepackage{xcolor}

% -------- enumerated sub-labels (a), (b), … --
\usepackage{enumitem}
\setlist[enumerate,1]{label=(\alph*),ref=\alph*}
% ---------------------------------------------

\usepackage{hyperref}
\hypersetup{hidelinks,
	colorlinks=true,
	allcolors=black,
	pdfstartview=Fit,
	breaklinks=true}
\usepackage{csquotes}
\usepackage{natbib}
\bibliographystyle{apalike}
\newtheorem{definition}{Definition}
\newtheorem{theorem}{Theorem}
\newtheorem{proposition}[theorem]{Proposition}
\newtheorem{lemma}[theorem]{Lemma}
\newtheorem{corollary}[theorem]{Corollary}
\newtheorem*{remark}{Remark}
\newtheorem{example}{Example}
\newtheorem{exercise}{Exercise}
\newtheorem{assumption}{Assumption}[section] % number within sections


\begin{document}

\begin{center}
    ECON 3123: Macroeconomic Theory I\\
    {\large \textbf{Tutorial Note 5: Asset Pricing}}\\
    Solution to Exercises\\
    Teaching Assistant: Harlly Zhou
\end{center}

\begin{enumerate}[label=\arabic*.]
    \item $\frac{1}{1+2\%}\left[\$1000 + \frac{\$1000}{1+3\%} + \frac{\$1000}{(1+3\%)^2} + \frac{\$1000}{(1+3\%)^3} + \frac{\$1000}{(1+3\%)^4}\right] = \$4662$.

    \item False. An upward-sloping yield curve typically signals expectations of continued growth and possibly rising interest rates in the future, not recession. By contrast, a downward-sloping (inverted) yield curve is the classic predictor of recession, because it reflects expectations that short-term rates will fall as the central bank cuts rates in response to weaker economic activity.

    \item (a) Very little will happen to stock prices. The present value discount factor for year one will decrease and the stock price will fall slightly.
    
    (b) Now all the discount factors get slightly smaller and the present value of all expected dividends falls resulting in a lower stock price.
    
    (c) The change in stock price will depend on the expected changes in magnitude of future output and future dividends relative to the change in interest rates.

    \item (a) Houses last a long time. Rents are likely to rise with inflation. Real interest rates would be better.
    
    (b) Let $R^e_{t+j}$ be the expected real rent on the house. Let $Q_{Ht}$ be the price of a house. Let $x_H$ be the risk premium on a house. The equation would be
    \[Q_{Ht} = \frac{R^e_{t+1}}{1 + r_{t+1} + x_H} + \frac{R^e_{t+2}}{(1 + r_{t+1} + x_H)(1 + r_{t+2} + x_H)} + \cdots.\]

    (c) The future rents would be discounted less and the price today would rise.

    (d) $x_H$ would decline in value. The discount factors would be less and the price would rise.
    
\end{enumerate}


\end{document}