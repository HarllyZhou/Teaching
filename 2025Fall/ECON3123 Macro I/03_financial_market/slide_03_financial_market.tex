\documentclass[xcolor=dvipsnames]{beamer}

\usepackage{etex}
\usepackage[utf8]{inputenc}
\usepackage{tgtermes}
\usefonttheme{default}
\renewcommand{\familydefault}{\rmdefault}
\usetheme{Madrid}
\usepackage{lmodern}

% --- Autumn + Ethereum palette (keep your structure/names) ---
\definecolor{autumnyellow}{HTML}{FFB000} % golden yellow (primary accent)
\definecolor{maplered}{HTML}{C1121F}   % maple red
\definecolor{chestnutbrown}{HTML}{8B5E3C}     % chestnutbrown
\definecolor{canvasbg}{HTML}{FAF4E6}    % warm paper background
\definecolor{graphite}{HTML}{3C3C3D}    % neutral text
\colorlet{autumndark}{autumnyellow!80!black}
\colorlet{canvasdark}{canvasbg!80!black}

% --- Beamer colors (same structure as yours) ---
\setbeamercolor{structure}{fg=autumnyellow}
\setbeamercolor{background canvas}{bg=canvasbg}
\setbeamercolor{frametitle}{bg=autumnyellow, fg=white}
\setbeamercolor{title}{bg=autumnyellow, fg=white}
\setbeamercolor{item}{fg=autumnyellow}
\setbeamercolor{section in toc}{fg=maplered}
\setbeamercolor{subsection in toc}{fg=autumnyellow}

% Optional: readable body text color
\setbeamercolor{normal text}{fg=graphite,bg=canvasbg}

\usepackage{tikz}
\usetikzlibrary{shapes.geometric, arrows.meta, positioning}
\usepackage{graphicx} % For \resizebox

% --- TikZ styles (same names; colors mapped to autumn/ETH) ---
\tikzset{
  block/.style = {rectangle, draw, fill=autumnyellow!40, text width=4cm, align=center, rounded corners, minimum height=2.0em},
  sideblock/.style = {rectangle, draw, fill=chestnutbrown!40, text width=4cm, align=center, rounded corners, minimum height=2.0em},
  decision/.style = {diamond, draw, fill=canvasbg!40, text width=5cm, align=center, aspect=2, inner sep=1pt},
  arrow/.style = {thick, ->, >=Stealth},
  mechanismblock/.style = {rectangle, draw, fill=maplered!40, text width=4cm, align=center, minimum height=2.5em},
  greydashedarrow/.style = {thick, dashed, ->, >=Stealth, draw=gray},
  blusharrow/.style = {thick, ->, >=Latex, draw=maplered},
  blank/.style = {rectangle, fill=maplered!20, text width=4cm, align=center, minimum height=2.0em}
}

%<---- do not use enumitem, does not work well with Beamer

\usepackage{amsmath, amssymb, amsfonts}
\usepackage{booktabs, array, dcolumn}
\usepackage{graphicx, subfig}
\usepackage{epstopdf}
\def\pdfshellescape{1}

% Icons/bullets
\setbeamertemplate{itemize items}[ball]
\setbeamertemplate{itemize subitem}[triangle]

% Theorems
\usepackage{amsthm}
\newtheorem{proposition}{Proposition}
\newtheorem*{quiz}{Quiz}
\newtheorem{claim}[proposition]{Claim}
\newtheorem{exercise}[proposition]{Exercise}
\newtheorem{remark}[proposition]{Remark}

% Links: use Ethereum blue
\definecolor{links}{HTML}{627EEA}
\hypersetup{colorlinks,linkcolor=,urlcolor=links}

\setbeamertemplate{navigation symbols}{}
\newcommand{\hilight}[1]{\colorbox{yellow}{#1}}

\DeclareMathOperator*{\plim}{plim}
\DeclareMathOperator*{\argmax}{arg\,max}
\DeclareMathOperator*{\E}{E}
\DeclareMathOperator*{\Var}{Var}
\DeclareMathOperator*{\Cov}{Cov}
\DeclareMathOperator*{\Corr}{Corr}
\DeclareMathOperator*{\supp}{supp}

\newcommand{\ind}{\mathrel{\perp \! \! \! \perp}}
\def\citeapos#1{\citeauthor{#1}'s (\citeyear{#1})}

\title[Measurment of Macroeconomy]{Tutorial 3: Investment and Financial Market}
\subtitle{ECON 3123: Macroeconomic Theory I}
\author[Harlly Zhou]{Harlly Zhou}
\institute[HKUST]{Department of Economics\\
HKUST Business School}
\date{}

\begin{document}

%%%%%%%%%%%%%%%%%%%%%%%%%%%%%%%%%%%%%%%%%%%%
%%%%%%%%%%%%%%%%%%%%%%%%%%%%%%%%%%%%%%%%%%%%
\begin{frame}
\titlepage
\end{frame}

\begin{frame}{Example 1: Wealth vs Income}
    Suppose that a person's wealth is \$50,000 and that her yearly income is \$60,000. Also suppose that her money demand function is given by
	\[M^d = \$ Y (0.35 - i).\]

	\begin{enumerate}
		\item Derive the demand for bonds. Suppose that the interest rate increases by 10 percentage points. What is the effect on her demand for bonds?
		\item What are the effects of an increase in wealth on her demand for money and her demand for bonds? Explain in words.
		\item What are the effects of an increase in income on her demand for money and her demand for bonds? Explain in words.
		\item Consider the statement ``When people earn more money, they obviously will hold more bonds.'' What is wrong with this statement?
	\end{enumerate}
\end{frame}

\begin{frame}{Example 2: Zero Lower Bound}
    Consider the following money demand function where $Y$ is the nominal income:
	\begin{align*}
		M^d = Y (0.91-5i).
	\end{align*}
	
	\begin{enumerate}
		\item Suppose that $Y=100$. If the central bank would like to target an interest rate of $2.2\%$, then what should be the money supply?
		\item If the nominal income increases to $Y=120$, then how should the central bank change its money supply to maintain the target interest rate?
		\item Keep $Y=100$. What is the largest value of the money supply at which the interest rate is positive?
		\item Once the interest rate is zero, can the central bank continue increasing the money supply?
	\end{enumerate}
\end{frame}


\end{document}