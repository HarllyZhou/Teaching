\documentclass[12pt]{article}

\usepackage[utf8]{inputenc}
\usepackage{geometry}
\geometry{a4paper,scale=0.75}
\linespread{1.5}
\usepackage{graphicx} 
\usepackage{float} 
\usepackage{subfig} 
\usepackage{enumerate}
\usepackage{enumitem}
\usepackage{amsmath}
\usepackage{array}
\usepackage{booktabs}
\usepackage{multirow}
\usepackage{amsfonts}
\usepackage[english]{babel}
\usepackage{amsthm}
\usepackage{dcolumn}
\usepackage{multicol}
\usepackage{stfloats}
\usepackage{lscape}
\usepackage[figuresright]{rotating}
\RequirePackage{pdflscape}
\usepackage[toc,page]{appendix}
\usepackage{geometry}
\usepackage{longtable}
\usepackage{comment}
\usepackage{xcolor}

% -------- enumerated sub-labels (a), (b), … --
\usepackage{enumitem}
\setlist[enumerate,1]{label=(\alph*),ref=\alph*}
% ---------------------------------------------

\usepackage{hyperref}
\hypersetup{hidelinks,
	colorlinks=true,
	allcolors=black,
	pdfstartview=Fit,
	breaklinks=true}
\usepackage{csquotes}
\usepackage{natbib}
\bibliographystyle{apalike}
\newtheorem{definition}{Definition}
\newtheorem{theorem}{Theorem}
\newtheorem{proposition}[theorem]{Proposition}
\newtheorem{lemma}[theorem]{Lemma}
\newtheorem{corollary}[theorem]{Corollary}
\newtheorem*{remark}{Remark}
\newtheorem{example}{Example}
\newtheorem{exercise}{Exercise}
\newtheorem{assumption}{Assumption}[section] % number within sections


\begin{document}

\begin{center}
    ECON 3123: Macroeconomic Theory I\\
    {\large \textbf{Tutorial Note 4: IS-LM Framework}}\\
    Solution to Exercises\\
    Teaching Assistant: Harlly Zhou
\end{center}

\begin{enumerate}[label=\arabic*.]
    \item (a) $Y = 840 - 2000 i$.
    
    (b) $Y = 780$.

    (c) $M/P = 1440$.

    (d) $C=304$; $I=276$; $G=200$; $C+I+G=780$.

    (e) $Y = 795$; $C =308.5$; $I = 286.5$. The increase in the money supply decreases the interest rate. Consumption and investment increase because output increases and interest rates decrease.

    (f) At the initial rate of 3\%, $Y$ equals 980 when $G$ is increased to 300. A fiscal expansion increases output. Consumption increases ($C = 364$) because output increases. When the central bank keeps interest rates at 3\% then investment increases ($I = 316$) as output increases.
    \item (1) B. (2) D. (3) B.
    \item (a) Same as Figure 6-4.
    
    (b) The value of the bank's capital falls to 10. The leverage ratio is 9.
    
    (c) If the deposits of the bank are insured by the government, then the health of the bank is irrelevant to the depositors. So, there is no need to withdraw funds from the bank.
    
    (d) The balance sheet is identical to the one in part (a), except that short-term credit replaces checking deposits
    
    (e) If lenders are nervous about the solvency of the bank, they will not be willing to provide short-term credit to the bank at low interest rates.
    
    (f) The bank must sell assets. If many banks are in this position and selling the same kind of assets, the value of these assets will fall. This will worsen the value of bank capital and make lenders more nervous.
    \item (1) No. Because the fiscal psace is limited, it is dangerous to continue to increase government spending. If so, the government will be much less able to support the economy and
    in huge amount of debt and it is likely to cause a crisis.
    
    (2) No. Increasing money supply moves the LM curve downwards. Therefore, ZLB will be hit at some point. At that time,
    there wil be no way to increase (stabalize) the
    output again.

    (3) Unconventional monetary policies. \textit{e.g.,} Quantitative easing; Forward guiding.
    \item (a) The risk premium is likely to fall. The IS curve will shift to the right. This will increase output and can be thought of as a sort of macroeconomic policy.
    
    (b) The risk premium is likely to fall. The IS curve would shift right and output would increase. Quantitative easing becomes a policy option when the nominal policy interest rate (the federal funds rate) is zero.

    (c) Strictly speaking, the increase in expected inflation does not directly affect the level of the real policy rate EXCEPT when the nominal policy rate remains constant. In Figure 6-9, this is the exact situation. The nominal policy rate of interest is zero and the real policy rate of interest is the negative of the expected inflation rate. Thus, if an action by the Fed increases expected inflation, this would decrease the real policy interest rate and shift the LM curve down. You would move ALONG the IS curve and output would rise.
\end{enumerate}






\end{document}