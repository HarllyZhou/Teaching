\documentclass[12pt]{article}

\usepackage[utf8]{inputenc}
\usepackage{geometry}
\geometry{a4paper,scale=0.75}
\linespread{1.5}
\usepackage{graphicx} 
\usepackage{float} 
\usepackage{subfig} 
\usepackage{enumerate}
\usepackage{enumitem}
\usepackage{amsmath}
\usepackage{array}
\usepackage{booktabs}
\usepackage{multirow}
\usepackage{amsfonts}
\usepackage[english]{babel}
\usepackage{amsthm}
\usepackage{dcolumn}
\usepackage{multicol}
\usepackage{stfloats}
\usepackage{lscape}
\usepackage[figuresright]{rotating}
\RequirePackage{pdflscape}
\usepackage[toc,page]{appendix}
\usepackage{geometry}
\usepackage{longtable}
\usepackage{comment}
\usepackage{xcolor}

% -------- enumerated sub-labels (a), (b), … --
\usepackage{enumitem}
\setlist[enumerate,1]{label=(\alph*),ref=\alph*}
% ---------------------------------------------

\usepackage{hyperref}
\hypersetup{hidelinks,
	colorlinks=true,
	allcolors=black,
	pdfstartview=Fit,
	breaklinks=true}
\usepackage{csquotes}
\usepackage{natbib}
\bibliographystyle{apalike}
\newtheorem{definition}{Definition}
\newtheorem{theorem}{Theorem}
\newtheorem{proposition}[theorem]{Proposition}
\newtheorem{lemma}[theorem]{Lemma}
\newtheorem{corollary}[theorem]{Corollary}
\newtheorem*{remark}{Remark}
\newtheorem{example}{Example}
\newtheorem{exercise}{Exercise}
\newtheorem{assumption}{Assumption}[section] % number within sections


\begin{document}

\begin{center}
    ECON 3123: Macroeconomic Theory I\\
    {\large \textbf{Problem Set 1 Solution}}\\
    Total Score: 50 points
\end{center}

\paragraph{Question 1 (9 points)}
\begin{enumerate}[label=\alph*.]
	\item (3 points, 1 point each for each correct result)
	\begin{align*}
	    Y_{\text{nominal}, 2009} &= 10 \times \$2,000 + 4 \times \$1,000 + 1,000 \times \$1 = \$25,000\\
		Y_{\text{nominal}, 2010} &= 12 \times \$3,000 + 6 \times \$500 + 1,000 \times \$1 = \$40,000\\
		g_{\text{nominal}} &= 60\%
	\end{align*}
	\item (3 points, 1 point each for each correct result)
	\begin{align*}
	    Y_{\text{real}, 2009} &= 10 \times \$2,500 + 4 \times \$750 + 1,000 \times \$1 = \$29,000\\
		Y_{\text{real}, 2010} &= 12 \times \$2,500 + 6 \times \$750 + 1,000 \times \$1 = \$35,500\\
		g_{\text{real}} &= 22.41\%
	\end{align*}
	\item (3 points, 1 point for each correct result)
	\begin{align*}
	    \text{deflator}_{2009} &= \frac{25,000}{29,000} = 0.86\\
		\text{deflator}_{2010} &= \frac{40,000}{35,500} = 1.13\\
		\pi_{\text{deflator}} &= 31.40\% \,\text{if you use 0.86 and 1.13, or }\\
		&\,\,\,\,\,\,\,\,30.70\%\, \text{if you use 25/29 and 400/355}
	\end{align*}
\end{enumerate}

\begin{remark}
	Normally I do not push hard on rounding issue. However, it will be good if you can keep \textbf{4 decimals} for real numbers, which means \textbf{2 decimals} for percentage expressions. For example, 0.3140 = 31.40\%.

	Moreover, no worries for different results in questions like part c. As long as your deflator is right and your inflation is consistent with your deflator, you will get full mark.
\end{remark}

\paragraph{Question 2 (19 points)}
\begin{enumerate}[label=\alph*.]
	\item (3 points, 1 point deduction for any missing step)
	
	The demand for goods and services is
	\begin{align*}
		Z &= C + I + G \qquad\qquad \text{(1 point)}\\
		&= (c_0 + b_0) + (c_1 + b_1)Y + (G - c_1 T).
	\end{align*}
	At equilibrium, supply equals demand, \textit{i.e.,}
	\begin{align*}
		Y = Z. \qquad\qquad \text{(1 point)}
	\end{align*}
	Solving the equation, we have
	\begin{align*}
		Y = \frac{1}{1-c_1-b_1}[c_0 + b_0 + G - c_1T]. \qquad\qquad \text{(1 point)}
	\end{align*}
	\item (4 points)
	
	The multiplier is $\frac{1}{1-c_1-b_1}$. (1 point)
	
	Including the $b_1Y$ term in the investment equation increases the multiplier. Increases in autonomous spending now creates a multiplier effect through two channels: consumption and investment. (2 points) 
	
	For the multiplier to be positive, the condition $c_1 + b_1 < 1$ is required.(1 point)
	\item (7 points, 1 bonus point)
	
	When $c_1 + b_1$ is greater than one, there is no multiplier effect. (1 point) When marginal propensity of spending exceeds one, the formula is nonsensical, that is, when $c_1 + b_1$ is greater than one, the multiplier is negative, which does not make sense. (1 point) 
	
	Another way of looking at the concept is that saving must equal investment in a closed economy. (2 point, as long as one mentions this IS relation arguement.)
	\begin{itemize}
		\item First, if $c_1 + b_1 > 1$, then the spending will be larger than income for sure, which will finally lead to a negative private saving. This is not sustainable. (2 points)
		\item Second, mathematically,
		\begin{align*}
			S &= Y - C - G = -c_0 + (1-c_1)Y + c_1T - G\\
			I &= b_0 + b_1 Y.
		\end{align*}
		Since $c_1 + b_1 > 1$, $b_1 > 1-c_1$. The investment increases faster than the saving. If $b_0 > -c_0 + c_1T - G$, that is, the intercept of $I$ is higher than that of $S$, then there will be no positive $Y$ such that $I=S$. (1 point, as long as one mentions some correct math. Bonus 1 point if the math makes sense as shown.)
	\end{itemize}
	In conclusion, in a closed economy, $c_1 + b_1$ can never be greater than one. 

	\item (5 points)
	
	Denote the change of a variable $x$ to be $\Delta x$. Since $\Delta b_0 >0$, 
	\begin{align*}
		\Delta Y &= \frac{\Delta b_0}{1-c_1-b_1} > 0 \qquad\qquad \text{(1 point)}\\
		\Delta I &= \Delta b_0 + b_1 \Delta Y > \Delta b_0. \qquad\qquad \text{(1 point)}
	\end{align*}
	The change in business confidence leads to an increase in output, which induces an additional increase in investment. (1 point)
	
	Since investment increases, and saving equals investment, saving must also increase. The increase in output leads to an increase in saving. (2 points. One must point out the IS relation, otherwise 1 point deduction.)
\end{enumerate}

\paragraph{Question 3 (9 points)}
\begin{enumerate}[label=\alph*.]
	\item (1 point)
	
	Output will fall.
	\item (5 points)
	
	Since output falls, investment will also fall. (1 point) 
	
	Public saving will not change. (1 point) 
	
	Private saving will fall, since investment falls, and investment equals saving. (2 points. One must point out the IS relation, otherwise 1 point deduction.)
	
	Since output and consumer confidence fall, consumption will also fall. (1 points)
	\item (3 points, 1 point each)
	
	Output, investment, and private saving would have risen.
\end{enumerate}

\paragraph{Question 4 (13 points)}
\begin{enumerate}[label=\alph*.]
	\item (3 points)
	
	Central bank money demand has two components: currency and reserves. Since all money is in checking accounts, demand for central bank money equals demand for reserves. (2 points)
	
	Therefore, demand for central bank money is
	\begin{align*}
		H^d = R^d = 0.1 M^d = 0.1 (\$ Y) (0.8 - 4i). \qquad\qquad \text{(1 point)}
	\end{align*}

	\item (2 points)
	Note that
	\begin{align*}
		H^S &= \$ 100 B\\
		H^d &= 0.1 (\$ 5,000 B) (0.8 - 4i).
	\end{align*}
	Letting $H^S = H^d$ (1 point, writing out the actual equation is okay. If one writes $M$, then no point.), we obtain $i = 15\%$. (1 point)

	\item (4 points)
	
	Since the public holds no currency, the money multiplier is $\frac{1}{0.1}=10$. (2 points)

	Therefore, the overall money supply is
	\begin{align*}
		M = 10 H = \$1,000 B. \qquad\qquad \text{(1 point)}
	\end{align*}
	Let $i=15\%$. We obtain $M^d = \$1,000 B = M$. (1 point)

	\item (2 points)
	
	Repeat the process in part (b). If H increases to \$300B the interest rate falls to 5\%.

	\item (2 points)
	
	The interest rate falls to 5\%, since when $H$ equals \$300B, M=(10)\$300B=\$3,000B.
\end{enumerate}

\end{document}