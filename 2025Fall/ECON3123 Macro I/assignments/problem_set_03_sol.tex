\documentclass[12pt]{article}

\usepackage[utf8]{inputenc}
\usepackage{geometry}
\geometry{a4paper,scale=0.75}
\linespread{1.5}
\usepackage{graphicx} 
\usepackage{float} 
\usepackage{subfig} 
\usepackage{enumerate}
\usepackage{enumitem}
\usepackage{amsmath}
\usepackage{array}
\usepackage{booktabs}
\usepackage{multirow}
\usepackage{amsfonts}
\usepackage[english]{babel}
\usepackage{amsthm}
\usepackage{dcolumn}
\usepackage{multicol}
\usepackage{stfloats}
\usepackage{lscape}
\usepackage[figuresright]{rotating}
\RequirePackage{pdflscape}
\usepackage[toc,page]{appendix}
\usepackage{geometry}
\usepackage{longtable}
\usepackage{comment}
\usepackage{xcolor}

% -------- enumerated sub-labels (a), (b), … --
\usepackage{enumitem}
\setlist[enumerate,1]{label=(\alph*),ref=\alph*}
% ---------------------------------------------

\usepackage{hyperref}
\hypersetup{hidelinks,
	colorlinks=true,
	allcolors=black,
	pdfstartview=Fit,
	breaklinks=true}
\usepackage{csquotes}
\usepackage{natbib}
\bibliographystyle{apalike}
\newtheorem{definition}{Definition}
\newtheorem{theorem}{Theorem}
\newtheorem{proposition}[theorem]{Proposition}
\newtheorem{lemma}[theorem]{Lemma}
\newtheorem{corollary}[theorem]{Corollary}
\newtheorem*{remark}{Remark}
\newtheorem{example}{Example}
\newtheorem{exercise}{Exercise}
\newtheorem{assumption}{Assumption}[section] % number within sections


\begin{document}

\begin{center}
    ECON 3123: Macroeconomic Theory I\\
    {\large \textbf{Problem Set 3 Solution}}\\
    Total Score: 100 points
\end{center}

\paragraph{Question 1 (12 points)}
\begin{enumerate}[label=(\alph*)]
    \item (4 points, 2 points for steps) $\pi_t - \pi_t^e = 0.1-2u_t$.
    \item (4 points) At medium run equilibrium, $\pi_t^e = \pi_t$ (2 points). Therefore, $u_n = 5\%$.
    \item (4 points) $Y_n = AL(1-u_n) = 190$.
\end{enumerate}

\paragraph{Question 2 (22 points)}
\begin{enumerate}[label=(\alph*)]
    \item (4 points) 4\%, 6\$, 8\%, 10\%.
    \item (5 points) At medium run equilibrium, $\pi_t^e = \pi_t$ (1 point). Therefore, $u_n = 5\%$ (2 points). The inflation rates will be both 10\% (2 points).
    \item (2 pointw) 9\%.
    \item (3 points) 4\% for $t, t+1, t+2, t+3$. 2\% for $t+4, t+5$. $u_{t+6} = 5\%$.
    \item (8 points) The Phillips curve implies that the current inflation is inversely related to unemployment rate. To cut the high inflation, the government needs to induce a recession so that unemployment rate increases (4 points for the fundamental relation). Moreover, since during the 1970s, the inflation expectation was deanchored. Thereofre, the government needs a stable low inflation to reanchor the inflation expectation (4 points for inflation expectation).
\end{enumerate}

\paragraph{Question 3 (28 points)}
\begin{enumerate}[label=(\alph*)]
    \item (12 points) Label axes ($Y(2), r/i, \pi-\bar{\pi}$, 3 points). Label curves ($IS, LM, PC$, 3 points). Label points and intercepts ($A(2), O(2), Y_A/Y_n(2), r_n/r^*$, 4 points). Match between the two graphs (2 points)
    \item (14 points) Label new $IS'$ curve (rightward) and the direction, new intersection $B$ (2), new output $Y_B$ (2), new inflation deviation $\pi'-\bar{\pi}$, and the movement along the $PC$ curve (upward) (1 point each, 1 more point for matching). $Y \uparrow, C \uparrow, I \uparrow, \pi \uparrow, u \downarrow, i \uparrow, r -$ (1 point each).
    \item (12 points) Label new $LM''$ curve (upward) and the direction, new intersection $C$, and the movement along the $PC$ curve (downward) (1 point each, 1 more point for matching). $Y -, C -, I \downarrow, \pi -, u -, i \uparrow, r \uparrow$ (1 point each).
\end{enumerate}

\paragraph{Question 4 (38 points)}
\begin{enumerate}[label=(\alph*)]
    \item (8 points) Correct data item and correct time span.
    \item (10 points) Scatter plots (2 points each, including labels), linearity (1 point each), trend line (1 point each), equation (1 point each).
    \item (10 points) Scatter plots (2 points each, including labels), linearity (1 point each), trend line (1 point each), equation (1 point each).
    \item (10 points) Scatter plots (2 points each, including labels), trend line (1 point each), equation (1 point each). Matching (1 point each).
\end{enumerate}


\end{document}