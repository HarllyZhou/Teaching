\documentclass[12pt]{article}

\usepackage[utf8]{inputenc}
\usepackage{geometry}
\geometry{a4paper,scale=0.75}
\linespread{1.5}
\usepackage{graphicx} 
\usepackage{float} 
\usepackage{subfig} 
\usepackage{enumerate}
\usepackage{enumitem}
\usepackage{amsmath}
\usepackage{array}
\usepackage{booktabs}
\usepackage{multirow}
\usepackage{amsfonts}
\usepackage[english]{babel}
\usepackage{amsthm}
\usepackage{dcolumn}
\usepackage{multicol}
\usepackage{stfloats}
\usepackage{lscape}
\usepackage[figuresright]{rotating}
\RequirePackage{pdflscape}
\usepackage[toc,page]{appendix}
\usepackage{geometry}
\usepackage{longtable}
\usepackage{comment}
\usepackage{xcolor}

% -------- enumerated sub-labels (a), (b), … --
\usepackage{enumitem}
\setlist[enumerate,1]{label=(\alph*),ref=\alph*}
% ---------------------------------------------

\usepackage{hyperref}
\hypersetup{hidelinks,
	colorlinks=true,
	allcolors=black,
	pdfstartview=Fit,
	breaklinks=true}
\usepackage{csquotes}
\usepackage{natbib}
\bibliographystyle{apalike}
\newtheorem{definition}{Definition}
\newtheorem{theorem}{Theorem}
\newtheorem{proposition}[theorem]{Proposition}
\newtheorem{lemma}[theorem]{Lemma}
\newtheorem{corollary}[theorem]{Corollary}
\newtheorem*{remark}{Remark}
\newtheorem{example}{Example}
\newtheorem{exercise}{Exercise}
\newtheorem{assumption}{Assumption}[section] % number within sections


\begin{document}

\begin{center}
    ECON 3123: Macroeconomic Theory I\\
    {\large \textbf{Problem Set 2 Solution}}\\
    Total Score: 50 points
\end{center}

\paragraph{Question 1 (7 points)}
\begin{enumerate}[label=\alph*.]
    \item (3 points) The $IS$ curve shifts left. The value of the parameter $c_0$ falls. Output falls.
    
    [1 point for conclusion (output falls). In the diagram, 1 point for correct shift of $IS$ curve, 1 point for notation.]
    \item (4 points) The level of consumption falls with the fall in output. There is a lower level of output and a lower level of $c_0$. Both factors lead to a decline in consumption. (1 point)
    
    Investment will also fall. Investment depends on output and the interest rate. The interest rate is unchanged. Output falls. Investment must fall. (1 point)

    If there was no change in $G$ or $T$, then the same argument is made, the decline in private investment must be accompanied by a decline in private saving. (2 points, 1 point for no change in $G$ and $T$, 1 point for $IS$ relation)
\end{enumerate}

\paragraph{Question 2 (10 points)}
\begin{enumerate}[label=\alph*.]
    \item (5 points) The $IS$ curve shifts left (2 points plus graph).  Output falls at the same interest rate (1 point). Investment depends positively on the level of output and negatively on the interest rate. The interest rate remains the same (1 point). Output falls. So, investment falls (1 point).
    \item (2 points) $Y = \frac{1}{1-c_1-b_1}[(c_0+b_0+G-c_1T)-b_2\bar{i}]$. (1 point deduction if bad step)
    \item (1 point) $I = \left[b_0 + \frac{b_1}{1-c_1-b_1}(c_0+b_0+G-c_1T)\right]+\frac{(1-c_1)b_2}{1-c_1-b_1}\bar{i}$.
    \item (2 points) $\frac{M}{P} = \frac{d_1}{1-c_1-b_1}(c_0+b_0+G-c_1T)-\left(\frac{b_2d_1}{1-c_1-b_1}-d_2\right)\bar{i}$. It increases with government spending.
\end{enumerate}

\paragraph{Question 3 (6 points, 2 point each)}
\begin{enumerate}[label=\alph*.]
    \item 7.72\%.
    \item 5.74\%.
    \item 4.15\%.
\end{enumerate}

\paragraph{Question 4 (9 points)}
\begin{enumerate}[label=\alph*.]
    \item (2 points) $Q_t=\$50,000$ when $r=5\%$. $Q_t=\$20,000$ when $r=8\%$. (1 point for correct formula but wrong results)
    \item (2 points) $Q_t=\$10,000$ when $r=5\%$. $Q_t=\$7,692.31$ when $r=8\%$. (1 point for correct formula but wrong results)
    \item (2 points) $Q_t=\$16,66.67$ when $r=5\%$. $Q_t=\$11,111.11$ when $r=8\%$. (1 point for correct formula but wrong results)
    \item (3 points) If the risk premium falls unexpectedly, the required return (discount rate) declines. (1 point) A lower discount rate increases the present value of the same future cash flows. (1 point) Therefore, the price of the stock rises immediately, even if expected dividends do not change. (1 point)
\end{enumerate}

\paragraph{Question 5 (18 points)}
\begin{enumerate}[label=\alph*.]
    \item (5 points, 1 point for graph) The fall in $G$ and the increase in $T$ shift the IS curve to the left. (1 point) If the Federal Reserve did not change interest rates, Y would fall. (1 point) Thus, to keep output at the same level, the Federal Reserve must cut interest rates and the LM curve will shift down. (1 point) Investment will increase since output remains the same and interest rates are lower.(1 point)
    \item (3 points) Receipts rose, outlays fell, and the budget deficit fell.
    \item (4 points) Data by effort. Monetary policy became more expansionary from 1995 to 1998.
    \item (2 points) In real terms, investment was 11.9\% of GDP in 1992 and increased every year over the period to reach 17.6\% of GDP in 2000.
    \item (4 points) Over the period 1993-2000, the average annual growth rate of GDP per person was 2.6\%. Over the period first four years of the period, the average annual growth rate was 2\%; over the second four years, the average annual growth rate was 3.2\%. The postwar average growth rate of income per person in the United States (1946 to 2018) was about 2\%. (Series for annual real GDP per capita in FRED data base is A939RX0Q048SBEA - 1947 value 14118; 2018 value 56912 measured in chained 2012 dollars.)
\end{enumerate}
\end{document}