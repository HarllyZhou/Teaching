\documentclass[12pt]{article}

\usepackage[utf8]{inputenc}
\usepackage{geometry}
\geometry{a4paper,scale=0.75}
\linespread{1.5}
\usepackage{graphicx} 
\usepackage{float} 
\usepackage{subfig} 
\usepackage{enumerate}
\usepackage{enumitem}
\usepackage{amsmath}
\usepackage{array}
\usepackage{booktabs}
\usepackage{multirow}
\usepackage{amsfonts}
\usepackage[english]{babel}
\usepackage{amsthm}
\usepackage{dcolumn}
\usepackage{multicol}
\usepackage{stfloats}
\usepackage{lscape}
\usepackage[figuresright]{rotating}
\RequirePackage{pdflscape}
\usepackage[toc,page]{appendix}
\usepackage{geometry}
\usepackage{longtable}
\usepackage{comment}
\usepackage{xcolor}

% -------- enumerated sub-labels (a), (b), … --
\usepackage{enumitem}
\setlist[enumerate,1]{label=(\alph*),ref=\alph*}
% ---------------------------------------------

\usepackage{hyperref}
\hypersetup{hidelinks,
	colorlinks=true,
	allcolors=black,
	pdfstartview=Fit,
	breaklinks=true}
\usepackage{csquotes}
\usepackage{natbib}
\bibliographystyle{apalike}
\newtheorem{definition}{Definition}
\newtheorem{theorem}{Theorem}
\newtheorem{proposition}[theorem]{Proposition}
\newtheorem{lemma}[theorem]{Lemma}
\newtheorem{corollary}[theorem]{Corollary}
\newtheorem*{remark}{Remark}
\newtheorem{example}{Example}
\newtheorem{exercise}{Exercise}
\newtheorem{assumption}{Assumption}[section] % number within sections


\begin{document}

\begin{center}
    ECON 3123: Macroeconomic Theory I\\
    {\large \textbf{Tutorial Note 7: Labour Market and Phillips Curve}}\\
    Teaching Assistant: Harlly Zhou
\end{center}

\subsection*{A Simple Model of the Labour Market}
\paragraph{Price Determination} Consider a production function
\[Y = AN.\]
The \textbf{marginal product of labour} (MPL) is $\frac{\partial Y}{\partial N} = A$. Suppose that the cost of hiring an extra worker is $W$. Then the marginal cost of production is $\frac{W}{A}$. Let $m$ be the markup (due to monopolistic power). Then the price level will be
\[ P = (1+m)\frac{W}{A}.\]

\paragraph{Wage Determination} Assume that the nominal wage is
\[W = A P^e F(u,z),\]
where $A$ is the MPL, $P^e$ is the expected price level, and $F$ is a function decreasing in unemployment rate $u$, and increasing in $z$, a variable capturing all other factors.

\paragraph{Natural Rate of Unemployment} From the price determination equation, we have
\[\frac{W}{P} = \frac{A}{1+m}.\]
From the wage determination equation, \textbf{since $P^e=P$ in the medium run}, we have
\[\frac{W}{P} = AF(u,z).\]
In a $\left(u, \frac{W}{P}\right)$ diagram, the pricing curve is a horizontal line, and the wage curve is a downward-sloping curve, as is shown in Figure \ref{fig:wu_01}. The equilibrium unemployment rate that is reached by the two curves is called the \textbf{natural rate of unemployment}.

\begin{figure}[htp]
    \centering
    \includegraphics[width=0.6\textwidth]{wu_01.png}
    \caption{Natural Rate of Unemployment}
    \label{fig:wu_01}
\end{figure}

\begin{example}
    Consider a linear production function
    \[Y = AN.\]
    Let the wage setting equation be
    \[\frac{W}{P^e} = A F(u,z),\]
    where $z$ is the union power and
    \[F(u,z) = 1 - \alpha u_t + \phi z_t.\]
    Suppose that the union can bargain for a real minimum wage at
    \[\frac{W}{P}\geq\underline{w}_t = \omega z_t\]
    \begin{enumerate}[label=(\arabic*)]
        \item When the minimum wage is not hit, find out the equilibrium real wage and write the natural rate of unemployment (not by log approximation) as a function of $z_t$ and parameters.
        \vspace{60pt}
        \item Suppose that the constraint binds. Find the natural rate of unemployment as a function of $z_t$ and parameters.
        \vspace{60pt}
        \item Find the threshold for $z_t$ where the minimum wage constraint binds. Denote it as $\bar{z}$.
        \vspace{60pt}
        \item When the constraint binds, how does the natural rate of unemployment change with union power? Explain the economic intuition.
        \vspace{60pt}
    \end{enumerate}
\end{example}


\begin{example}
    Consider the following production function
    \[Y = A \log N.\]
    The wage determination equation, in the medium run equilibrium, is
    \[\frac{W}{P} = AF(u,z)\equiv A(1-\alpha u_t + z).\]
    \begin{enumerate}[label=(\arabic*)]
        \item Let $m$ be the markup. Write the price setting equation, \textit{i.e.,} write $P$ as a function of $N$.
        \vspace{60pt}
        \item Let the population $L$ be normalized to 1 and assume that it is constant. Find a quadratic equation characterizing the natural rate of unemployment. Which solution to the equation you should keep? Why?
        \vspace{60pt}
        \item Suppose that the labour productivity $A$ increases. Using a diagram, explain what happens to the new equilibrium real wage and the new natural rate of unemployment.
        \vspace{60pt}
    \end{enumerate}
\end{example}
\begin{exercise}
    Chapter 7, Question 5 in Blanchard, Olivier (2021), \textit{Macroeconomics}, 8th ed., Pearson.
\end{exercise}

\subsection*{Deriving the Phillips Curve}
Recall that the labour market equilibrium is the intersection of
\begin{align*}
    \text{Wage setting: }\,\,&\frac{W}{P^e} = AF(u,z)\\
    \text{Price setting: }\,\,&P = (1+m)\frac{W}{A}.
\end{align*}
Combining the two curves, we obtain
\[P = P^e(1+m) F(u,z).\]
Assume that
\[F(u,z) = 1-\alpha u +z.\]
Then
\[P_t = P^e_t(1+m) (1-\alpha u_t +z).\]
Dividing both sides by $P_{t-1}$, we obtain
\[\log\frac{P_t}{P_{t-1}} = \log\frac{P^e_t}{P_{t-1}} + \log (1+m) + \log (1-\alpha u_t +z).\]
By approximation, we have the \textbf{Phillips curve}:
\[\pi_t = \pi^e_t + (m+z) - \alpha u_t.\]

\paragraph{Original Phillips Curve}
Assume that inflation does not persist: $\pi^e = \bar{\pi}$,meaning that past inflation rates are not informative to predict new inflation. Then the Phillips curve becomes
\[\pi_t = (\bar{\pi}+m+z) - \alpha u_t.\]
Denote $\beta := \bar{\pi}+m+z$. Then we have
\[\pi_t = \beta - \alpha u_t.\]
However, there are periods when this original Phillips curve is not supported by the data.

\paragraph{Accelerationist Phillips Curve}
Assume that inflation persists: $\pi^e = \pi_{t-1}$, meaning that people use the inflation from the past period as the indicator of expected inflation. Then the Phillips curve becomes
\[\pi_t = \pi_{t-1} + (m+z) - \alpha u_t.\]
Denote $\gamma := m+z$. Then we have
\[\Delta \pi_t = \gamma - \alpha u_t.\]
The period now fits.

\paragraph{Inflation Expectations}
As in the previous two versions of PC, we have different methods to form expectations for inflation. Combining the two methods, we can assume that
\[\pi_t = (1-\theta)\bar{\pi} + \theta\pi_{t-1}.\]

\begin{example}
    Continue with Example 1.
    \begin{enumerate}[label=(\arabic*)]
        \item Derive the Phillips curve.
        \vspace{60pt}
        \item Write the accelerationist version of the Phillips curve. Compare it with
        \[\Delta \pi_t = \gamma - \alpha u_t.\]
        What is the difference? Where does it come from? How do they differ economically?
        \vspace{60pt}
    \end{enumerate}
\end{example}

\begin{exercise}
    Chapter 8, Question 5 in Blanchard, Olivier (2021), \textit{Macroeconomics}, 8th ed., Pearson.
\end{exercise}

\subsection*{Natural Rate of Unemployment Revisited}
Recall the the natural rate of unemployment is the rate at which $P=P^e$, \textit{i.e.}, the labour market is at medium-run equilibrium. Since $P=P^e$, we also have $\pi=\pi^e$ in medium-run equilibrium. Then the PC becomes
\[0 = (m+z)-\alpha u_n,\]
which yields
\[u_n = \frac{m+z}{\alpha}.\]
Then the PC (not at equilibrium) can be rewritten as
\[\pi_t - \pi_t^e = -\alpha (u_t - u_n).\]

\paragraph{Wage Indexation}
\textbf{Wage indexation} is a provision that automatically increases wages in line with inflation. Suppose that there is a $\lambda$ proportion of indexed contracts. Then we have two types of wage settings:
\begin{align*}
    \text{Indexed Wage: }\, & W_1 = A P_t F(u_t,z)\\
    \text{Normal Wage: }\, & W_2 = A P^e_t F(u_t,z).
\end{align*}
Then the price setting becomes
\[P_t = (1+m)\frac{\lambda W_1 + (1-\lambda) W_2}{A} = [\lambda P_t + (1-\lambda)P_t^e] F(u,z)(1+m),\]
which yields the following PC curve:
\[\pi_t = [\lambda\pi_t + (1-\lambda)\pi_t^e]-\alpha(u_t-u_n).\]
Rearranging the terms, we get
\[\pi_t-\pi_t^e = -\frac{\alpha}{1-\lambda}(u_t-u_n).\]

\begin{example}
    Continue with example 1 and 2.
    \begin{enumerate}[label=(\arabic*)]
        \item Find the natural rate of unemployment using the Phillips curve. Check your answer with Example 1 part (2). Do they match? Rewrite the Phillips curve using the natural rate of unemployment.
        \vspace{60pt}
        \item Suppose that now half of the contracts are indexed. Write the Phillips curve equation.
        \vspace{60pt}
        \item What is the effect of wage indexation on the relation between inflation and unemployment?
        \vspace{60pt}
    \end{enumerate}
\end{example}

\begin{exercise}
    Chapter 8, Question 6 in Blanchard, Olivier (2021), \textit{Macroeconomics}, 8th ed., Pearson.
\end{exercise}


\end{document}