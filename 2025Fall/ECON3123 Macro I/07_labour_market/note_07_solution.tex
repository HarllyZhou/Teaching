\documentclass[12pt]{article}

\usepackage[utf8]{inputenc}
\usepackage{geometry}
\geometry{a4paper,scale=0.75}
\linespread{1.5}
\usepackage{graphicx} 
\usepackage{float} 
\usepackage{subfig} 
\usepackage{enumerate}
\usepackage{enumitem}
\usepackage{amsmath}
\usepackage{array}
\usepackage{booktabs}
\usepackage{multirow}
\usepackage{amsfonts}
\usepackage[english]{babel}
\usepackage{amsthm}
\usepackage{dcolumn}
\usepackage{multicol}
\usepackage{stfloats}
\usepackage{lscape}
\usepackage[figuresright]{rotating}
\RequirePackage{pdflscape}
\usepackage[toc,page]{appendix}
\usepackage{geometry}
\usepackage{longtable}
\usepackage{comment}
\usepackage{xcolor}

% -------- enumerated sub-labels (a), (b), … --
\usepackage{enumitem}
\setlist[enumerate,1]{label=(\alph*),ref=\alph*}
% ---------------------------------------------

\usepackage{hyperref}
\hypersetup{hidelinks,
	colorlinks=true,
	allcolors=black,
	pdfstartview=Fit,
	breaklinks=true}
\usepackage{csquotes}
\usepackage{natbib}
\bibliographystyle{apalike}
\newtheorem{definition}{Definition}
\newtheorem{theorem}{Theorem}
\newtheorem{proposition}[theorem]{Proposition}
\newtheorem{lemma}[theorem]{Lemma}
\newtheorem{corollary}[theorem]{Corollary}
\newtheorem*{remark}{Remark}
\newtheorem{example}{Example}
\newtheorem{exercise}{Exercise}
\newtheorem{assumption}{Assumption}[section] % number within sections


\begin{document}

\begin{center}
    ECON 3123: Macroeconomic Theory I\\
    {\large \textbf{Tutorial Note 7: Labour Market and Phillips Curve}}\\
    Solution to Exercises\\
    Teaching Assistant: Harlly Zhou
\end{center}

\begin{enumerate}[label=\arabic*.]
    \item \begin{enumerate}[label=(\alph*)]
        \item The computer network administrator has more bargaining power. She is much harder to replace.
        \item The rate of unemployment is the most important indicator of labor market conditions. When the rate of unemployment increases, it becomes easier for firms to find replacements, and worker bargaining power falls.
        \item In our model, the real wage is always given by the price-setting relation: 
        \[\frac{W}{P} = \frac{1}{1+\mu}.\]
        Since the price-setting relation depends on the actual price level and not the expected one, this relation holds in the short run and the medium run of our model.
    \end{enumerate}
    \item \begin{enumerate}[label=(\alph*)]
        \item $u_n=0.1/2=.05$.
        
        In this case in all periods since $\pi_t^e = \bar{\pi}$ in all periods since $\theta=0$.

        Initial unemployment is 0.05 or 5\%. In period $t$, unemployment is reduced to 3\%. If we then use Phillips curve, inflation in period $t$ is:
        \[\bar{\pi} + 0.1 - (2 \times 0.03) = +.04 = 0.06.\]
         Given the model we have, this will also be the value of inflation in period $t+1$, $t+2$, $t+3$, $t+5$. This value of inflation is a higher value than the anchored rate of inflation $\bar{\pi}$.
         \item This does not make much sense. Every period actual inflation exceeds expected anchored inflation by 4\%. Remember that $\bar{\pi} = \pi^e$ in this model.
         \item This will put more weight on previous year's inflation in forming the expectation of inflation. In the periods from $t+1$ to $t+5$, a reasonable person might think last period's inflation (4 percentage points higher than $\bar{\pi}$) is a better predictor of actual inflation than the fixed value $\bar{\pi}$.
         \item The values will be
         \begin{align*}
            & t+6: \text{solving }\, \pi_t = \pi_t^e + 0.1 - 2u_t \text{and using }\, \pi_t^e = \pi_{t-1}.\\
            & t+6\,\, u_{t+6} = .03, \pi_{t+6} = \bar{\pi} + .04 + 0.1 - (2\times .03) = \bar{\pi} + 0.08\\
            & t+7\,\, u_{t+7} = .03, \pi_{t+7} = \bar{\pi} + .08 + 0.1 - (2\times .03) = \bar{\pi} + 0.12\\
            & t+8\,\, u_{t+8} = .03, \pi_{t+8} = \bar{\pi} + .12 + 0.1 - (2\times .03) = \bar{\pi} + 0.16.
         \end{align*}
         \item You can see that keeping unemployment be low the natural rate leads to an ever accelerating rate of inflation when $\theta = 1$. Hence the other name for the natural rate is the NAIRU, the non-accelerating inflation rate of unemployment. In this case inflation rises by 4 percentage points each year. This does not seem to be a feasible long run policy choice.
         \item If the unemployment rate is at the natural rate of unemployment (5\%) and we assume that $\theta = 1$ then we solve $\pi_t - \pi_{t-1} = 0.1 - (2 \times 0.05) = 0$. In this situation, in every period actual inflation equals the previous period's rate of inflation. Inflation does not change.
    \end{enumerate}
    \item \begin{enumerate}[label=(\alph*)]
        \item This will move the model of expected inflation so that
        \begin{align*}
            \pi_t &= \pi_{t-1} - 2(u_t - 0.05) = \pi_{t-1} + 2\% = 2\%.\\
            \pi_t &= 2\%,\\
            \pi_{t+1} &= 4\%,\\
            \pi_{t+2} &= 6\%,\\
            \pi_{t+3} &= 8\%.
        \end{align*}
        \item $\pi_t = 0.5 \pi_t + 0.5 \pi_{t-1} - 2(u_t - 0.05)$, or $\pi_t = \pi_{t-1} - 4(u_t-0.05)$.
        \item $\pi_t = 4\%, \pi_{t+1} = 8\%, \pi_{t+2} = 12\%, \pi_{t+3} = 16\%$.
        \item As indexation increases, inflation becomes more sensitive to the difference between the unemployment rate and the natural rate.
    \end{enumerate}
\end{enumerate}

\end{document}