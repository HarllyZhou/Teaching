\documentclass[12pt]{article}

\usepackage[utf8]{inputenc}
\usepackage{geometry}
\geometry{a4paper,scale=0.75}
\linespread{1.5}
\usepackage{graphicx} 
\usepackage{float} 
\usepackage{subfig} 
\usepackage{enumerate}
\usepackage{enumitem}
\usepackage{amsmath}
\usepackage{array}
\usepackage{booktabs}
\usepackage{multirow}
\usepackage{amsfonts}
\usepackage[english]{babel}
\usepackage{amsthm}
\usepackage{dcolumn}
\usepackage{multicol}
\usepackage{stfloats}
\usepackage{lscape}
\usepackage[figuresright]{rotating}
\RequirePackage{pdflscape}
\usepackage[toc,page]{appendix}
\usepackage{geometry}
\usepackage{longtable}
\usepackage{comment}
\usepackage{xcolor}

% -------- enumerated sub-labels (a), (b), … --
\usepackage{enumitem}
\setlist[enumerate,1]{label=(\alph*),ref=\alph*}
% ---------------------------------------------

\usepackage{hyperref}
\hypersetup{hidelinks,
	colorlinks=true,
	allcolors=black,
	pdfstartview=Fit,
	breaklinks=true}
\usepackage{csquotes}
\usepackage{natbib}
\bibliographystyle{apalike}
\newtheorem{definition}{Definition}
\newtheorem{theorem}{Theorem}
\newtheorem{proposition}[theorem]{Proposition}
\newtheorem{lemma}[theorem]{Lemma}
\newtheorem{corollary}[theorem]{Corollary}
\newtheorem*{remark}{Remark}
\newtheorem{example}{Example}
\newtheorem{exercise}{Exercise}
\numberwithin{equation}{section}
\newtheorem{assumption}{Assumption}[section] % number within sections

\begin{document}

\begin{center}
    ECON 3123: Macroeconomic Theory I\\
    {\large \textbf{Tutorial Note 1: Measurement of Macroeconomy}}\\
    Solution to Exercises\\
    Teaching Assistant: Harlly Zhou
\end{center}

\begin{enumerate}[label=\arabic*.]
    \item \$5 million. Note that the \$10 million paid for computers is not part of value added. Note also that the fact that the firm produces an intermediate good doesn't mean that it doesn't contribute to GDP.
    \item \$87,000 - \$11,000 paid for intermediate goods + \$1000 change in inventories = \$77,000.
    \item Inventories are valued at the cost of production, so the 1000 canoes in inventory were valued at \$1000 each, for a total of \$1 million. Four thousand canoes at \$1250 each totaled \$5 million. Salmon as a final good were worth \$27 million (the other \$3 million were used up as an intermediate good), and corn worth \$55 million was grown. So total GDP (in millions) was \$1 + \$5 + \$27 + \$55 = \$88 million. 
    \item (1) D. It is investment. (2) C. Government transfer is not counted into $G$.
    \item D. Chain index enforces consistency in growth rates, not levels.
    \item 4.5\%.
    \item \begin{enumerate}[label=\alph*.]
        \item The cost of the consumer price basket in 2017 = JP\yen 63,860 + JP\yen 320,000 = JP\yen 383,860
        \item 2010: JP\yen 370,800; 2010: JP\yen 369,700; 2012: JP\yen 369,700; 2013: JP\yen 371,130; 2014: JP\yen 380,980; 2015: JP\yen 384,000; 2016: JP\yen 383,500; 2017: JP\yen 383,860
        \item CPI: 99.70, 99.68, 100.09, 102.75, 103.56, 103.43, 103,52
        \item Inflation: -0.3\%, 0.0\%, 0.4\%, 2.7\%, 0.8\%, -0.1\%, 0.1\%
        \item Inflation was negative in 2011 and in 2016. These negative rates relate to the decrease in economic growth of Japan during the period. Since December 2010, Japan's economic policy (the ``Abenomics'' named after the elected Prime Minister, Shinzo Abe) focuses on boosting economic growth and increasing inflation.
        \item Due to the decline in the international price of oil since 2014, inflation decreased in 2014 and 2015, and consumers could have increased their consumption of other products. However, in Japan, private consumption of goods decreased in 2014 and 2015. This means that consumption is not only determined by the level of price but by several other factors, notably by consumer's expectations, which will be discussed further in subsequent chapters of the book.
        \item In 2010, a household was able to buy one basket of goods and services with JP¥ 371,000. In 2017, with the same amount, a household was able to buy less than one basket (0.96 baskets) because the purchasing power of money declined by 3.6\% during this period. To ensure that their purchasing power does not decline, households can negotiate with their employers to increase wages by at least by 3.6\%. This is why many trade unions and employers index wages and salaries to inflation.
        \item The Bank of Japan has adopted a technique called inflation targeting, which is usually used to control inflation and keep prices stable. However, in the framework of Japan's Abenomics policy, the bank set the target for the annual inflation rate of the CPI at 2\% not to limit inflation but to target the level of inflation, as moderate inflation is considered to be a condition for growth.
        \item Core inflation rate: -0.8\%, -0.4\%, -0.2\%, 0.7\%, 0.9\%, 0.1\%, 0.9\%. Core inflation is lower than the CPI in general. During the period 2010-2017, energy and food prices declined. The prices of other items in the basket of goods and services must have declined more than oil and agricultural produce prices have.
        \end{enumerate}
    \item D. This is a consequence of CPI overstatement, not the cause.
    \item (1) 5.6\%; (2) 1.5\%; (3) $[(8867.1/8516.3) - 1]\times100\% = 4.1\%$.
\end{enumerate}
\end{document}